\documentclass[12pt]{article}
 \usepackage[margin=1in]{geometry}
\usepackage{amsmath,amsthm,amssymb,amsfonts, enumitem, fancyhdr, color, comment, graphicx, environ, fancyhdr}
\pagestyle{fancy}
\usepackage{cmap}
\usepackage[T2A]{fontenc}
\usepackage[utf8]{inputenc}
\usepackage[english,russian]{babel}
\setlength{\parindent}{5ex}
\setlength{\headheight}{35pt}
\theoremstyle{definition}
\newtheorem{definition}{Definition}[section]
% \lhead{Stochastic differential equations}
% \rhead{Kirill Zakharov}
\title{Стохастические дифференциальные уравнения}
\author{Кирилл Захаров}
\date{2020}
\begin{document}
\maketitle
\tableofcontents
\section{Предварительные сведения}
Пусть $(\Omega,\mathcal{F},P)$ - вероятностное пространство.$\xi=(\xi_1,...,\xi_n)$ - случайная последовательность.
Пусть $\theta_k\xi=(\xi_{k+1},\xi_{k+2},...)$.
\begin{definition}
  $\xi $ называется стационарной в узком смысле, если $\forall k\geq1$ распределения $\theta_k\xi$ и $\xi$ совпадают
  $P((\xi_1,\xi_2,...)\in B)=P((\xi_{k+1},\xi_{k+2},...)\in B), \: B\in\mathcal{B}(\mathbb{R}^{\infty})$.
\end{definition}
\begin{definition}
  Пусть $ T$ - множество значений параметра $t$. Случайным процессом назовем параметризованный набор случайных величин
   $\{ \xi_t\}_{t\in T}$, принимающих значения
  в $\mathbb{R}^n$.
\end{definition}
\section{Уравнение Ито}
\subsection{Винеровский процесс}
Рассмотрим дискретную модель блуждания.
\begin{equation}
  x=x_0+\mu_0 n+\sigma_0\sqrt{n}\cdot\varepsilon
\end{equation}
Накопленное стохастическое изменение пропорционально нормальному распределению с параметрами 0 и 1, т.е.
 $\varepsilon_1+...+\varepsilon_n=\sqrt{n}\cdot\varepsilon$, где $\varepsilon\sim N(0,1)$.
Пусть $\Delta t$ - длительность одного шага, тогда количество шагов в момент $t-t_0$ равно $n=\dfrac{t-t_0}{\Delta t}$.
Пусть $\sigma^2=\sigma_0^2/\Delta t$ и $\mu=\dfrac{\mu_0}{\Delta t}$. Получим следующее уравнение.
\begin{equation}
  x(t)=x(t_0)+\mu(t-t_0)+\sigma\sqrt{t-t_0}\cdot\varepsilon
\end{equation}
Ширина процесса $x(t)$ увеличивается со временем пропорционально корню $\sqrt{t-t_0}$, а максимум сдвигается со скоростью
 $\mu$.
 Рассмотрим изменение $dx=x(t)-x(t_0)$ за бесконечно малый интервал $dt=t-t_0$. Тогда получим
 \begin{equation}
   dx=\mu dt+\sigma \sqrt{dt}\cdot\varepsilon
 \end{equation}
 Обозначим $ \sqrt{dt}\cdot\varepsilon$ за $\delta W$. Данный процесс называется непрерывным винеровским процессом.
\begin{equation}
  dx=\mu dt+\sigma \delta W
\end{equation}
\subsection{Процесс Ито}
Общие процессы Ито представляют собой "деформацию" простого винеровского блуждания при помощи функций
$a(x,t)$ и $b(x,t)$.
\begin{equation}
  dx=a(x,t) dt+b(x,t) \delta W
\end{equation}
где $a(x,t)$-коэффициент сноса, $b(x,t)$ - коэффициент волатильности, $\delta W $ - бесконечно малый винеровский шум.
Также $b^2(x,t)$ называют диффузией. Для моделирования процесса Ито воспользуемся формулой в конечно-разностном представлении.
\begin{equation}
  x_{k+1}=x_k+a(x_k,t_k)\Delta t+b(x_k,t_k)\sqrt{\Delta t}\cdot\varepsilon_k
\end{equation}
В произвольный фиксированный момент времени $x(t)$ - это случайная величина, свойства которой определяются при помощи
 $\varepsilon$ и значения $t$.
 \subsection{Лемма Ито}
 \begin{equation}
   dF=\bigg(\frac{\partial F}{\partial t}+a(x,t)\frac{\partial F}{\partial x}+
   \frac{b^2(x,t)}{2}\frac{\partial^2 F}{\partial x^2}\bigg)dt+
   b(x,t)\frac{\partial F}{\partial x}\delta W
 \end{equation}
Слагаемое перед $dt$ обозначим за $f(t)$, а перед $\delta W$ за $s(t)$
Необходимо подобрать $F(x, t)$ так, чтобы функции $f(t), s(t)$ были зависимы только от $t$.
В результате получим следующие выражения.
\begin{align}
  &\frac{\partial F}{\partial x}=\frac{s(t)}{b(x,t)}\\
  &\frac{\partial F}{\partial t}+s(t)\bigg(\frac{a(x,t)}{b(x,t)}-\frac{1}{2}\frac{\partial b(x,t)}{\partial x}\bigg)=f(t)
\end{align}
Возьмём частные производные уравнения (8) по $t$ и (9) по $x$. Вычитая их, получим условие совместности
\begin{equation}
 \frac{1}{s(t)}\frac{\partial}{\partial t}\bigg\{\frac{s(t)}{b(x,t)}\bigg\}=
 \frac{1}{2}\frac{\partial^2 b(x,t)}{\partial x^2}-\frac{\partial}{\partial x}\bigg\{\frac{a(x,t)}{b(x,t)}\bigg\}
\end{equation}
Если при данных $a(x,t)$ и $b(x,t)$ можно подобрать такую функцию $s(t)$,
 при которой уравнение (10) обратится в тождество, то получим решение стохастического уравнения (5) в следующей неявной форме
 \begin{equation}
  F(x,t)=F(x_0,t_0)+\int_{t_0}^tf(\tau)d\tau + \bigg(\int_{t_0}^ts^2(\tau)d\tau\bigg)^{1/2}\cdot\varepsilon
 \end{equation}

 \section{Простые стохастические модели}
 \subsection{Логарифмическое блуждание}
Данный процесс также называется геометрическим броуновским блужданием и определяется уравнением (7).
\begin{equation}
dx=\mu x dt +\sigma x \delta W
\end{equation}
где $\mu, \sigma=const$. Если стохастический член равен 0 $\sigma=0$, то получаем уравнение экспоненциального пространство
при $\mu>0$ и снижения при $\mu<0$.\\
Здесь $a(x,t)=\mu x$ и $b(x,t)=\sigma x$. Подставим их в условие совместности.
\begin{align}
  &\frac{1}{s(t)}\frac{\partial}{\partial t}\bigg\{ \frac{s(t)}{\sigma x} \bigg\}=\notag
  0-\frac{\partial \frac{\mu}{\sigma}}{\partial x}=0\\
  &\Rightarrow \dot{s}(t)=0 \Rightarrow s=const\notag
\end{align}
Так как $s(t)$ равна любой константе, удобно ее взять равной $\sigma$, для более простого нахождения $F(x,t)$.
Воспользуемся уравнением (8), проинтегрировав его. $\displaystyle \int \frac{\partial F}{\partial x}dx=
\int\frac{\sigma}{\sigma x}dx\Rightarrow F(x,t)=ln(x)$. Теперь, зная $F(x,t)$ легко найти $f(t)$ по формуле (9).
$\displaystyle f(t)=0+\sigma\bigg(\frac{\mu}{\sigma}-\frac{1}{2}\sigma\bigg)\Rightarrow f(t)=\mu-\sigma^2/2$.
Далее воспользуемся уравнением (11) при $t_0=0$.
\begin{align}
  &F(x,t)=ln(x_0)+\int_{t_0}^t (\mu-\sigma^2/2)d\tau+\bigg(\int_{t_0}^t \sigma^2d\tau\bigg)^{1/2}\cdot\varepsilon=\notag\\
  &=ln(x_0)+(\mu-\sigma^2/2)t+\sqrt{\sigma^2}\sqrt{t}\cdot\varepsilon\notag\\
  &\Rightarrow ln(x)=ln(x_0)+(\mu-\sigma^2/2)t+\sigma\sqrt{t}\cdot\varepsilon\notag\\
  &\Rightarrow x(t)=x_0\cdot e^{(\mu-\sigma^2/2)t+\sigma\sqrt{t}\cdot\varepsilon}\notag
\end{align}

\subsection{Процесс Орнштейна-Уленбека}
Данный процесс задается следующим уравнением.
\begin{equation}
dx=-\beta(x-\alpha) dt +\sigma\delta W
\end{equation}
где $\sigma=const,\beta>0$ - характеризуется величину силы притяжения к равновесному состоянию $\alpha$
Здесь $a(x,t)=-\beta(x-\alpha)$ и $b(x,t)=\sigma$. Подставим их в условие совместности.
\begin{align}
  &\frac{1}{s(t)}\frac{\partial}{\partial t}\bigg\{ \frac{s(t)}{\sigma} \bigg\}=\notag
  0-\frac{\partial \frac{-\beta(x-\alpha)}{\sigma}}{\partial x}\\
  &\Rightarrow \frac{1}{s(t)\sigma}\dot{s}(t)=\frac{\beta}{\sigma} \Rightarrow \dot{s}(t)=\beta s(t)\\
  &\Rightarrow \ln{s(t)}=\beta t\Rightarrow s(t)=C e^{\beta t}\notag
\end{align}
Пусть $\displaystyle C=\sigma\Rightarrow \frac{\partial F}{\partial x}=\frac{s(t)}{b(x,t)}=e^{\beta t}\Rightarrow
\int \frac{\partial F}{\partial x}dx=\int e^{\beta t} dx\Rightarrow F(x,t)=x\cdot e^{\beta t}+C_1$.
Теперь, зная $F(x,t)$ найдем $f(t)$ по формуле (9).
$\displaystyle f(t)=\beta x e^{\beta t}+\sigma e^{\beta t}\bigg(\frac{-\beta(x-\alpha)}{\sigma}-\frac{1}{2}\cdot 0\bigg)\Rightarrow
 f(t)=\alpha\beta e^{\beta t}$.
 Далее воспользуемся уравнением (11) при $t_0=0$.
 \begin{align}
   &F(x,t)=x_0e^{\beta t_0}+\int_{t_0}^t (\alpha\beta e^{\beta \tau})d\tau+\bigg(\int_{t_0}^t \sigma^2e^{2\beta \tau}d\tau\bigg)^{1/2}\cdot\varepsilon=\notag\\
   &=x_0e^{\beta t_0}+\alpha\beta e^{\beta t}-\alpha+\sqrt{\frac{\sigma^2}{2\beta}e^{2\beta t}-\frac{\sigma^2}{2\beta}}\cdot\varepsilon\notag\\
   & xe^{\beta t}=x_0+\alpha\beta e^{\beta t}-\alpha+\sqrt{\frac{\sigma^2}{2\beta}e^{2\beta t}-\frac{\sigma^2}{2\beta}}\cdot\varepsilon\\
   & x(t)=x_0 e^{-\beta t}+\alpha-\alpha e^{-\beta t}+\frac{\sigma}{\sqrt{2\beta}}e^{-\beta t}\cdot \sqrt{e^{2\beta t}-1}\cdot\varepsilon=\notag\\
   &\alpha+(x_0-\alpha)e^{-\beta t}+\frac{\sigma}{\sqrt{2 \beta}}\sqrt{1-e^{-2\beta t}}\cdot\varepsilon\notag
 \end{align}
 Если $\beta>0$, то среднее значение при $t\to\infty$ стремится к $\alpha$. Решение $x(t)$ находится в интервале шириной $2\dfrac{\sigma}{\sqrt{2\beta}}$. 
\end{document}
