\documentclass[12pt]{article}
 \usepackage[margin=1in]{geometry}
\usepackage{amsmath,amsthm,amssymb,amsfonts, enumitem, fancyhdr, color, comment, graphicx, environ, fancyhdr}
\pagestyle{fancy}
\usepackage{cmap}
\usepackage[T2A]{fontenc}
\usepackage[utf8]{inputenc}
\usepackage[english,russian]{babel}
\usepackage[table]{xcolor}

\setlength{\parindent}{0ex}
\setlength{\headheight}{35pt}

\lhead{Optimization theory}
\rhead{Kirill Zakharov}

\begin{document}
\section*{Нелинейная оптимизация}

\section*{Task 12.6}
\textbf{Условие}\\
Дана функция двух переменных $f(x_1, x_2)$.\\
a. Найти стационарную точку и вычислить в ней значение функции;\\
b. Определить экстремум, если он есть; \\
c. Проверить функцию на выпуклость/вогнутость.
\begin{equation}
  f = 18x_1-4x_1x_2+6x_2^2+16x_1-16x_2+2
\end{equation}
\textbf{Решение:}
\begin{enumerate}
  \item[a)] Найдем стационарную точку
  \begin{gather}
    \begin{cases}
        \dfrac{\partial f}{\partial x_1}=0\\
        \dfrac{\partial f}{\partial x_2}=0
    \end{cases};
    \begin{cases}
      36x_1-4x_2+16=0\\
      -4x_1+12x_2-16=0
    \end{cases};\\
    \begin{cases}
      x_2=9x_1+4\\
      -4x_1+12(9x_1+4)-16=0
    \end{cases};
    \begin{cases}
      x_2=\dfrac{16}{13}\\
      x_1=-\dfrac{4}{13}
    \end{cases}\notag
  \end{gather}
  $\Rightarrow f(x_1,x_2)=18\dfrac{16}{169}+4\dfrac{4}{13}\dfrac{16}{13}+6\dfrac{16^2}{169}-16\dfrac{4}{13}-16\dfrac{16}{13}+2=
  -\dfrac{134}{13}$
  \item[b)] Составим матрицу Гессе
  \begin{equation}
    H_f(x_1,x_2)=\begin{pmatrix}
      36&-4\\-4&12
    \end{pmatrix}
  \end{equation}
  $M_1(H)=36,M_2(H)=416\Rightarrow$ Матрица Гессе положительно определенная, т.к. все угловые миноры положительны. Значит стационарная
  точка $\big(-\dfrac{4}{13};\dfrac{16}{13}\big)$ является точкой минимума.
  \item[c)] Из тех же соображений функция является выпуклой вниз.
\end{enumerate}
\begin{center}
  \includegraphics[scale=0.6]{task12.png}
\end{center}

\section*{Task 13.6}
\textbf{Условие}\\
Дана функция трех переменных $f(x_1, x_2, x_3)$.\\
a. Найти стационарную точку и вычислить в ней значение функции;\\
b. Определить экстремум, если он есть; \\
c. Проверить функцию на выпуклость/вогнутость.
\begin{equation}
  f = 4x_1^2+3x_2^2+2x_3^2+4x_1x_2-2x_1x_3-4x_2x_3+2x_1-3x_2+1
\end{equation}
\textbf{Решение:}
\begin{enumerate}
  \item[a)] Найдем стационарную точку
  \begin{gather}
    \begin{cases}
        \dfrac{\partial f}{\partial x_1}=0\\
        \dfrac{\partial f}{\partial x_2}=0\\
        \dfrac{\partial f}{\partial x_3}=0
    \end{cases};
    \begin{cases}
      8x_1+4x_2-2x_3+2=0\\
      6x_2+4x_1-4x_3-3=0\\
      4x_3-2x_1-4x_2=0
    \end{cases};\\
    \begin{cases}
      16x_3-16x_2+4x_2-2x_3+2=0\\
      6x_2+8x_3-8x_2-4x_3-3=0\\
      x_1=2x_3-2x_2
    \end{cases};
    \begin{cases}
      14x_3-12(2x_3-\dfrac{3}{2})+2=0\\
      x_2=2x_3-\dfrac{3}{2}\\
      x_1=2x_3-2x_2
    \end{cases};\notag\\
    \begin{cases}
      x_3=2\\
      x_2=\dfrac{5}{2}\\
      x_1=-1
    \end{cases}\notag
  \end{gather}
  $\Rightarrow f(x_1,x_2,x_3)=
  -\dfrac{15}{4}$
  \item[b)] Составим матрицу Гессе
  \begin{equation}
    H_f(x_1,x_2,x_3)=\begin{pmatrix}
      8&4&-2\\4&6&-4\\-2&-4&4
    \end{pmatrix}
  \end{equation}
  $M_1(H)=8,M_2(H)=32,M_3(H)=40 \Rightarrow$ Матрица Гессе положительно определенная, т.к. все угловые миноры положительны. Значит стационарная
  точка $\big(-1;\dfrac{5}{2};2\big)$ является точкой минимума.
  \item[c)] Из тех же соображений функция является выпуклой вниз.
\end{enumerate}
\end{document}
