\documentclass[12pt]{article}
 \usepackage[left=3cm,right=3cm,top=2cm,bottom=3cm]{geometry}
\usepackage{amsmath,amsthm,amssymb,amsfonts, enumitem, fancyhdr, color, comment, graphicx, environ, fancyhdr}
\pagestyle{fancy}
\usepackage{cmap}
\usepackage[T2A]{fontenc}
\usepackage[utf8]{inputenc}
\usepackage[english,russian]{babel}
\setlength{\parindent}{5ex}
\setlength{\headheight}{35pt}

\lhead{Optimization theory}
\rhead{Kirill Zakharov}

\begin{document}
\section*{Task 2.6}

\textbf{Условие:}
\begin{gather}
2x_1-x_2\leqslant 8 \notag\\
-x_1+x_2\leqslant 1\\
-x_1+x_2\geqslant -1\notag\\
f=-5x_1-2x_2\to \min \notag
\end{gather}
Рассмотрим равенства и построим границы.
\begin{gather}
2x_1-x_2= 8 \notag\\
-x_1+x_2= 1\\
-x_1+x_2= -1\notag
\end{gather}

Чтобы найти минимальное значение функции, необходимо двигаться вдоль антиградиента.\\
$grad\, f = \{\frac{\partial f}{\partial x_1};\frac{\partial f}{\partial x_2}\}=\{-5, -2\}\\
\Rightarrow \overline{grad}\, f=-grad\, f=\{5, 2\}$\\

Далее нарисуем границы выпуклой многогранной области, зададим линию уровня $f(x_1,x_2)=C$ и
будем двигаться по направлению антиградиента, пока не окажемся в вершине области.
\begin{center}
\includegraphics[scale=0.65]{graphic_task2.png}\\
\end{center}

Таким образом минимальное решение есть точка пересечения двух ограничений $-x_1+x_2= 1 $ и $2x_1-x_2= 8 $.
Решая систему получим\\
\begin{equation}
  \begin{cases}
  -x_1+x_2= 1 \\
  2x_1-x_2= 8
\end{cases}
\end{equation}
$x_1=9;x_2=10\Rightarrow f = -65$

\section*{Task 3.6}
\begin{center}
  \begin{tabular}{|c|c|}
    \hline
    Каноническая форма:&\\
    $2x_1-x_2+y_1= 8$&$\begin{pmatrix}2&-1&1&0&0\\-1&1&0&1&0\\1&-1&0&0&1\end{pmatrix}
    \times\begin{pmatrix}x_1\\x_2\\y_1\\y_2\\y_3\end{pmatrix}
    =\begin{pmatrix}8\\1\\1\end{pmatrix} $\\
    $-x_1+x_2+y_2= 1$&\\
    $x_1-x_2+y_3= 1$&\\
    $f=-5x_1-2x_2\to\min$&$\begin{pmatrix}-5&-2\end{pmatrix}\times \begin{pmatrix}x_1\\x_2\end{pmatrix}\to\min$\\
    \hline
    Стандартная форма:&\\
    $-2x_1+x_2 \geqslant -8$&$\begin{pmatrix}-2&1\\1&-1\\-1&1\end{pmatrix}
    \times\begin{pmatrix}x_1\\x_2\end{pmatrix}
    \geqslant\begin{pmatrix}-8\\-1\\-1\end{pmatrix} $ \\
    $x_1-x_2 \geqslant -1$&\\
    $-x_1+x_2 \geqslant -1$&\\
    \hline
  \end{tabular}
\end{center}
$N=5; M=3\Rightarrow 3$ базисных переменных;$2$ свободные переменные.\\
В качестве базисных выберем $y_1,y_2,y_3$
\begin{align}
&y_1=-2x_1+x_2+8 \notag\\
&y_2=x_1-x_2+ 1\\
&y_3=-x_1+x_2+1 \notag\\
&f=-5x_1-2x_2\notag
\end{align}

1) $x_1=x_2=0;\,f=0;\,y_1=8;\,y_2=1;\,y_3=1$\\
Наибольшее убывание функции $f$ просиходит по переменной $x_1\Rightarrow x_1 \to \infty$
\begin{equation}
  \begin{cases}
  y_1=-2x_1+8=0\\
  y_2=x_1+1=0\\
  y_3=-x_1+1=0
\end{cases};
\begin{cases}
  x_1=4 \\
  x_1=-1 \\
  x_1=1
\end{cases}
\end{equation}
Выбираем наименьшее значение переменной $x_1$, ей соответсвует базисная переменная $y_3$. Меняем их местами
$x_1 \to y_3;y_3\to x_1$. Теперь $x_1$ - базисная переменная, а $y_3$ - свободная переменная.\\
$\Rightarrow$
\begin{align}
    &x_1=1-y_3+x_2 \notag \\
    &y_1=-2(1-y_3+x_2)+8+x_2=6+2y_3-x_2 \\
    &y_2=(1-y_3+x_2)+1+x_2=2-y_3\notag\\
    &f = -5(1-y_3+x_2)=-5+5y_3-7x_2\notag
\end{align}

2) $y_3=x_2=0;\,f=-5;\,x_1=1;\,y_1=6;\,y_2=2$\\
Функция $f$ убывает за счет переменной $x_2 \Rightarrow x_2 \to \infty$

\begin{equation}
  \begin{cases}
  x_1=1+x_2=0\\
  y_1=6-x_2=0\\
  y_2=2-y_3
\end{cases};
\begin{cases}
  x_2=-1 \\
  x_2=6
\end{cases}
\end{equation}

$x_2\to y_1;y_1\to x_2$
\begin{align}
  &x_2=6+2y_3-y_1\notag\\
  &x_1=1-y_3+(6+2y_3-y_1)=7+y_3-y_1\\
  &y_2=2-y_3\notag\\
  &f=-5+5y_3-7(6+2y_3-y_1)=-47-9y_3+7y_1\notag
\end{align}

3) $y_1=y_3=0;\,f=-47;\,x_1=7;\,x_2=6;\,y_2=2$\\
Теперь функция $f$ убывает по переменной $y_3\Rightarrow y_3\to \infty$
\begin{equation}
  \begin{cases}
  x_2=6+2y_3=0\\
  x_1=7+y_3=0\\
  y_2=2-y_3=0
\end{cases};
\begin{cases}
  y_3=-3 \\
  y_3=-7\\
  y_3=2
\end{cases}
\end{equation}
$\Rightarrow y_3 \to y_2;y_2\to y_3$
\begin{align}
  &y_3=2-y_2\notag\\
  &x_2=6+2(2-y_2)-y_1=10-2y_2-y_1\\
  &x_1=7+(2-y_2)-y_1=9-y_2-y_1\notag\\
  &f=-47-9(2-y_2)+7y_1=-65+9y_2+7y_1\notag
\end{align}

4) $y_1=y_2=0;\,f=-65;\,x_1=9;\,x_2=10;\,y_3=2$
\end{document}
