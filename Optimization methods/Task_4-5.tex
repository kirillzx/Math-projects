\documentclass[12pt]{article}
 \usepackage[margin=1in]{geometry}
\usepackage{amsmath,amsthm,amssymb,amsfonts, enumitem, fancyhdr, color, comment, graphicx, environ, fancyhdr}
\pagestyle{fancy}
\usepackage{cmap}
\usepackage[T2A]{fontenc}
\usepackage[utf8]{inputenc}
\usepackage[english,russian]{babel}
\setlength{\parindent}{5ex}
\setlength{\headheight}{35pt}

\lhead{Optimization theory}
\rhead{Kirill Zakharov}

\begin{document}
\section*{Task 4.6}

\textbf{Условие:}
\begin{align}
&x_1-8 x_2\leqslant 10 \notag\\
&x_1+x_2\geqslant 6\notag\\
&x_1\geqslant 2\\
&x_2 \leqslant 5\notag\\
&x_{1,2}\geqslant0\notag\\
&f=10x_1-6x_2\to \max \notag
\end{align}

\begin{center}
  \begin{tabular}{|c|c|}
    \hline
    Каноническая форма:&\\
    $x_1-8x_2+y_1= 10$&$\begin{pmatrix}1&-8&1&0&0&0\\1&1&0&-1&0&0\\1&0&0&0&-1&0\\0&1&0&0&0&1\end{pmatrix}
    \times\begin{pmatrix}x_1\\x_2\\y_1\\y_2\\y_3\\y_4\end{pmatrix}
    =\begin{pmatrix}10\\6\\2\\5\end{pmatrix} $\\
    $x_1+x_2-y_2= 6$&\\
    $x_1-y_3= 2$&\\
    $x_2+y_4=5$&\\
    $x_{1,2}\geqslant 0;\: y_j \geqslant 0$&\\
    $f=10x_1-6x_2\to\max$&$\begin{pmatrix}10&-6\end{pmatrix}\times \begin{pmatrix}x_1\\x_2\end{pmatrix}\to\max$\\
    \hline
    Стандартная форма:&\\
    $x_1-8x_2 \leqslant 10$&$\begin{pmatrix}1&-8\\-1&-1\\-1&0\\0&1\end{pmatrix}
    \times\begin{pmatrix}x_1\\x_2\end{pmatrix}
    \leqslant\begin{pmatrix}10\\-6\\-2\\5\end{pmatrix} $ \\
    $-x_1-x_2 \leqslant -6$&\\
    $-x_1 \leqslant -2$&\\
    $x_2 \leqslant 5$&\\
    $x_{1,2}\geqslant 0$&\\
    $f=10x_1-6x_2\to\max$&$\begin{pmatrix}10&-6\end{pmatrix}\times \begin{pmatrix}x_1\\x_2\end{pmatrix}\to\max$\\
    \hline
  \end{tabular}
\end{center}
\subsection*{1. Метод штрафа}
\begin{align}
  &x_1-8x_2+y_1=10\notag\\
  &x_1+x_2-y_2+R_1=6\notag\\
  &x_1-y_3+R_2=2\notag\\
  &x_2+y_4=5\\
  &x_{1,2}\geqslant 0;\: y_j\geqslant 0\notag\\
  &R_{1,2}\geqslant 0\notag\\
  &F=10x_1-6x_2-M*R_1-M*R_2\to\max\notag
\end{align}
\includegraphics[scale=0.7]{task4_1-2.png}\\
\includegraphics[scale=0.7]{task4_3-4.png}\\
\includegraphics[scale=0.7]{task4_5.png}\\
\begin{align}
  &f=-10y_1-74y_4-30R_1-30R_2+470\\
  &x_1=50;\:x_2=5;\:y_2=49;\:y_3=48;\:y_1=0;\:y_4=-45\notag
\end{align}
\subsection*{2. Двухэтапный метод}
\begin{align}
  &x_1-8x_2+y_1=10\notag\\
  &x_1+x_2-y_2+R_1=6\notag\\
  &x_1-y_3+R_2=2\notag\\
  &x_2+y_4=5\\
  &x_{1,2}\geqslant 0;\: y_j\geqslant 0;\: R_{1,2}\geqslant 0\notag\\
  &f=10x_1-6x_2\to\max\notag\\
  &R = R_1+R_2\to\min\notag
\end{align}
\subsubsection*{Первый этап}
\includegraphics[scale=0.7]{task4_2_1.png}\\
\subsubsection*{Второй этап}
\includegraphics[scale=0.7]{task4_2_2.png}\\
\includegraphics[scale=0.7]{task4_2_3.png}\\
\includegraphics[scale=0.7]{task4_2_4.png}\\
\begin{align}
  &f=-10y_1-74y_4+470\\
  &x_1=50;\:x_2=5;\:y_2=49;\:y_3=48;\:y_1=0;\:y_4=-45\notag
\end{align}
\section*{Task 5.6}
\subsection*{Task 2.6}
\textbf{Условие:}
\begin{align}
  &2x_1-x_2\leqslant8\notag\\
  &-x_1+x_2\leqslant1\\
  &-x_1+x_2\geqslant-1\notag\\
  &f=-5x_1-2x_2\to\min\notag
\end{align}
Каноническая форма
\begin{align}
  &2x_1-x_2+y_1=8\to\lambda_1\notag\\
  &-x_1+x_2+y_2=1\to\lambda_2\\
  &x_1-x_2+y_3=1\to\lambda_3\notag\\
  &f=-5x_1-2x_2\to\min\notag
\end{align}
Составим двойственную задачу
\begin{align}
 &\Phi=8\lambda_1+\lambda_2+\lambda_3\to\max\notag\\
 &x_1:\: 2\lambda_1 -\lambda_2+\lambda_3 \leqslant -5\notag\\
 &x_2:\: -\lambda_1 +\lambda_2-\lambda_3 \leqslant -2\\
 &y_1:\: \lambda_1 \leqslant 0\notag\\
 &y_2:\: \lambda_2 \leqslant 0\notag\\
 &y_3:\: \lambda_3 \leqslant 0\notag
\end{align}
Целевая функция в оптимуме имеет вид:
\begin{equation}
  f=-65+9y_2+7y_1
\end{equation}
Решение
\begin{align}
  -7=\lambda_1-0\Rightarrow \lambda_1=-7\notag\\
  -9=\lambda_2-0\Rightarrow \lambda_2=-9\\
  0=\lambda_3-0\Rightarrow \lambda_3=0\notag
\end{align}
\newpage
Проверка результата
\begin{center}\includegraphics[scale=1]{task5_1.png}\end{center}
% \subsection*{1.2. По стандартной форме}
% \begin{align}
%   &-2x_1+x_2\geqslant-8\to\lambda_1\notag\\
%   &x_1-x_2\geqslant-1\to\lambda_2\\
%   &-x_1+x_2\geqslant-1\to\lambda_3\notag\\
%   &f=-5x_1-2x_2\to\min\notag
% \end{align}
%
% Составим двойственную задачу
% \begin{align}
%  &\Phi=-8\lambda_1-\lambda_2-\lambda_3\to\max\notag\\
%  &x_1:\: -2\lambda_1 +\lambda_2-\lambda_3 \leqslant -5\\
%  &x_2:\: \lambda_1 -\lambda_2+\lambda_3 \leqslant -2\notag\\
%  &\lambda_{1,2,3}\geqslant0\notag
% \end{align}
\subsection*{Task 4.6}
\textbf{Условие:}
\begin{align}
  &x_1-8x_2\leqslant10\notag\\
  &x_1+x_2\geqslant6\notag\\
  &x_1\geqslant2\\
  &x_2\leqslant5\notag\\
  &f=10x_1-6x_2\to\max\notag
\end{align}
Каноническая форма
\begin{align}
  &x_1-8x_2+y_1=10\to\lambda_1\notag\\
  &x_1+x_2-y_2=6\to\lambda_2\notag\\
  &x_1-y_3=2\to\lambda_3\\
  &x_2+y_4=5\to\lambda_4\notag\\
  &f=10x_1-6x_2\to\max\notag
\end{align}
Целевая функция в оптимуме имеет вид:
\begin{equation}
  f=470-10y_1-74y_4
\end{equation}
Составим двойственную задачу
\begin{align}
 &\Phi=10\lambda_1+6\lambda_2+2\lambda_3+5\lambda_4\to\min\notag\\
 &x_1:\: \lambda_1 +\lambda_2+\lambda_3 \geqslant 10\notag\\
 &x_2:\: -8\lambda_1 +\lambda_2+\lambda_4 \geqslant -6\notag\\
 &y_1:\: \lambda_1 \geqslant 0\notag\\
 &y_2:\: \lambda_2 \leqslant 0\\
 &y_3:\: \lambda_3 \leqslant 0\\
 &y_4:\: \lambda_4 \geqslant 0\notag
 % &\Rightarrow \Phi=10\lambda_1+6(\lambda_2^{''}-\lambda_2^{'})+2(\lambda_3^{''}-\lambda_3^{'})
 % +5\lambda_4\to\min\notag\\
 % &x_1:\: \lambda_1 +\lambda_2^{''}-\lambda_2^{'}+\lambda_3^{''}-\lambda_3^{'} \geqslant 10\notag\\
 % &x_2:\: -8\lambda_1 +\lambda_2^{''}-\lambda_2^{'}+\lambda_4 \geqslant -6\notag
\end{align}
Решение
\begin{align}
  &10=\lambda_1-0=\lambda_1\notag\\
  &0=\lambda_2-0=\lambda_2\\
  &0=\lambda_3-0=\lambda_3\notag\\
  &74=\lambda_4-0=\lambda_4\notag
\end{align}
% \subsection*{2.2. По стандартной форме}
% \begin{align}
%   &x_1-8x_2\leqslant10\to\lambda_1\notag\\
%   &-x_1-x_2\leqslant-6\to\lambda_2\notag\\
%   &-x_1\leqslant-2\to\lambda_3\\
%   &x_2\leqslant5\to\lambda_4\notag\\
%   &f=10x_1-6x_2\to\max\notag
% \end{align}
% Составим двойственную задачу
% \begin{align}
%  &\Phi=10\lambda_1-6\lambda_2-2\lambda_3+5\lambda_4\to\min\notag\\
%  &x_1:\: \lambda_1 -\lambda_2-\lambda_3 \geqslant 10\\
%  &x_2:\: -8\lambda_1 -\lambda_2+\lambda_4 \geqslant -6\notag\\
%  &\lambda_{1,2,3,4}\geqslant0\notag
% \end{align}
\end{document}
