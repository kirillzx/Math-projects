\documentclass[12pt]{article}
 \usepackage[margin=1in]{geometry}
\usepackage{amsmath,amsthm,amssymb,amsfonts, enumitem, fancyhdr, color, comment, graphicx, environ, fancyhdr}
\pagestyle{fancy}
\usepackage{cmap}
\usepackage[T2A]{fontenc}
\usepackage[utf8]{inputenc}
\usepackage[english,russian]{babel}
\setlength{\parindent}{0ex}
\setlength{\headheight}{35pt}

\lhead{Optimization methods}
\rhead{Kirill Zakharov}

\newtheorem*{exersize}{Упражнение}

\newtheorem{lemma}{Лемма}[section]
\newtheorem{theorem}{Теорема}[section]
\newtheorem*{consequence}{Следствие}
\newtheorem*{statement}{Утверждение}
\newtheorem{property}{Свойство}
\newtheorem*{fact}{Fact}

\theoremstyle{definition}
\newtheorem{definition}{Определение}[section]
\newtheorem*{problem}{Problem}
\newtheorem*{example}{Пример}
\theoremstyle{remark}
\newtheorem*{remark}{Замечание}

\title{Методы оптимизации. Теоремы.}
\author{Кирилл Захаров}
\date{2021 г.}
\begin{document}
\maketitle
\tableofcontents

\section{Линейное программирование. Базисное решение, допустимое множество, оптимальное решение}

\begin{theorem}
  Множество допустимых решений есть выпуклое множество.
\end{theorem}
\begin{lemma}
  Базисные решения являются вершинами выпуклой многогранной области.
\end{lemma}
\begin{theorem}
  Оптимальное решение является базисным решением. (Оптимальное решение лежит в углах выпуклой многогранной области).
\end{theorem}
\section{Линейное программирование. Двойственная задача}

\begin{theorem}[Основное неравенство двойственности]
  Пусть заданы прямая задача $D: X\;f(X)$ и двойственная $\Omega :\Lambda\;\varphi(\Lambda)$. Тогда для любых допустимых планов прямой и двойственной задачи их целевые функции связаны неравенствами.
  \begin{align}
    &f(X)\to\min \Rightarrow f(X)\geqslant \varphi (\Lambda)\\
    &f(X)\to\max \Rightarrow f(X)\leqslant \varphi (\Lambda)\notag
  \end{align}
\end{theorem}
\begin{theorem}[Критерий оптимальности Канторович]
  Если на допустимых планах прямой  $X$ и двойственной задачи $\Lambda$ значения их целевых функций совпадают, то планы  $X$ и $\Lambda$ являются оптимальными и наоборот.
\end{theorem}

\begin{theorem}
Для существования оптимального плана как прямой, так и двойственной задач $\iff$ существование какого-либо допустимого плана для каждой из этих задач.
\end{theorem}

\begin{theorem}
Если прямая задача имеет оптимальное решение, то и двойственная имеет оптимальное решение.
\end{theorem}

\begin{theorem}
Если прямая задача не имеет решения из-за неограниченности целевой функции, то система ограничений двойственной задачи противоречива.
\end{theorem}

\begin{theorem}[О дополняющей нежесткости]
Необходимым и достаточным условием того, что прямая и двойственная задачи имеют оптимальное решение, является выполнение условий дополняющей нежесткости.
\begin{align}
  &\lambda_j\bigg(\sum_{i=1}^N a_{ji}x_i-b_j\bigg)=0\\
  &x_i\bigg(\sum_{j=1}^M a_{ji}\lambda_i-c_i\bigg)=0\notag
\end{align}
\end{theorem}

\section{Задача нелинейной безусловной оптимизации}

$x\in O\subseteq \mathbb{R}^N$
\begin{definition}
  $Y=(y_1,...,y_N)$ - точка локального минимума или максимума, если $\exists \;\varepsilon >0$, такое что выполняется
   \begin{align}
     &f(Y) \leqslant f(Y+\delta X)\text{ или }f(Y) \geqslant f(Y+\delta X)
   \end{align}
   для всех $\delta X=(\delta x_1,...,\delta x_N)\;\big|0<|\delta x_i|<\varepsilon$.
\end{definition}
\begin{definition}
 $Y$ - точка строгого экстремума, если неравенства выполняются строго.
\end{definition}
\begin{definition}
 $Y$ называется точкой глобального экстремума, если неравенства (3) выполняются во всей области.
\end{definition}
$\min{f(x)}=\max{-f(X)}$
\begin{definition}
Функция, имеющая единственный экстремум называется унимодальной.
\end{definition}
\begin{lemma}
  Если область допустимых значений, определяемая системой ограничений равенств, содержит точку $Y$ и ее окрестность, то $M<N$.\\
  \begin{equation}
    Y\subseteq D\land U_{\varepsilon}(Y)\subseteq D\Rightarrow M<N
  \end{equation}
\end{lemma}
\begin{lemma}
  Пусть область допустимых значений, определяемая системой ограничений равенств задачи на условный экстремум, содержит хотя бы одну точку $Y$. Если набор градиентов $grad\: \psi_j$ линейно независим и $rank\: J=M<N$, то $D$ вместе с каждой точкой $X$ содержит некоторую непустую ее окрестность.
\end{lemma}
\begin{theorem}
Пусть задана функция $f(x)$ и $x\in O=\mathbb{R}^1$. Если в точке $Y$ функция $f(x)$ имеет локальный экстремум, то $\displaystyle \frac{\partial f(Y)}{\partial x}=0$.
\end{theorem}
\begin{theorem}[Необходимое условие экстремума 1-го порядка]
Пусть задана функция $f(X)$ и $X\in O=\mathbb{R}^N$. Пусть $Y$ точка локального экстремума. Тогда $grad\:f(Y)=0$.
\end{theorem}

\begin{theorem}[Критерий Сильвестра]
  \item
\begin{enumerate}
  \item Матрица $A$ является положительно определенной $\iff$ когда все ее угловые миноры больше 0;
  \item Матрица $A$ является отрицательно определенной $\iff$ когда все ее угловые миноры образуют знакочередующийся ряд, начиная со знака <<$-$>>;
  \item Матрица $A$ является положительно полуопределенной $\iff$ $A$ вырождена и все ее главные миноры $m_i(A)\geqslant 0$;
  \item Матрица $A$ является отрицательно полуопределенной $\iff$ $m_i(A)=0$ или $sign\:m_i(A)=sign\:(-1)^i$.
\end{enumerate}
\end{theorem}
\begin{theorem}[Необходимое условие экстремума 2-го порядка]
Пусть задана функция $f(X)\;X\in\mathbb{R}^N$. Пусть  $f(X)$ дважды дифференцируема в окрестности точки $Y$. Тогда если $Y$ - точка локального минимума (максимума), то $H_f(Y)$ положительно полуопределенная (отрицательно полуопределенная).
\end{theorem}
\begin{theorem}[Достаточное условие экстремума 2-го порядка]
Пусть задана функция $f(X)\;X\in\mathbb{R}^N$. Пуста $f(X)$ имеет стационарную точку, в которой вторые частные производные существуют и непрерывны. Если $H_f(Y)$ положительно определена (отрицательно определена), то $Y$ точка минимума (максимума).
\end{theorem}
\begin{theorem}
  \item
  \begin{enumerate}
    \item Пусть $f(X)$ дифференцируема в точке $Y\in\mathbb{R}^N$. \\Тогда если $\delta X\in\mathbb{R}^N \:\big| grad\:f(Y)\cdot \delta X <(>)0
  \Rightarrow \delta X \in W_-(Y,f)(W_+(Y,f))$;
  \item Если $\delta X \in W_-(Y,f)(W_+(Y,f))$. Тогда  $grad\:f(Y)\cdot \delta X \leqslant(\geqslant)0$.
  \end{enumerate}
\end{theorem}
\section{Задача нелинейной условной оптимизации}
\begin{theorem}[Связь между $W_{+/-}(Y,f)$ и $V(Y,f)$]
Если точка $Y$ точка локального минимума (максимума), то $W_-(Y,f)\cap V(Y,f)= \varnothing \;(W_+(Y,f)\cap V(Y,f)= \varnothing) $.
\end{theorem}
\begin{theorem}[Вейерштрасс]
  Пусть $D$ - компакт и $f(X)$ непрерывная функция определенная на $D$.\\
  Тогда существует точка $Y$ глобального минимума (максимума).
\end{theorem}
\begin{theorem}[Необх. условие 1-го рода. Правило множителей Лагранжа]
  Пусть $Y\in D\subseteq \mathbb{R}^N$ - точка локального экстремума. Пусть $f(X),\psi_j(X)$ - непрерывно дифференцируемы и пусть
  в точке $Y\; J(Y)=\bigg\{\dfrac{\partial \psi_j(Y)}{\partial x_i}\bigg\}$ имеет ранг равный $M$.\\
  Тогда существуют неравные одновременно нулю вектор $\Lambda^{'}$ и $\lambda_0^{'}\;\big|$ точка $(\Lambda^{'},\lambda_0^{'},Y)$ - стационарная точка функции Лагранжа, т.е. $grad\:L(\Lambda^{'},\lambda_0^{'},Y)=0$
\end{theorem}
\begin{theorem}[Необх. условие 2-го рода]
  Пусть $Y\in D\subseteq \mathbb{R}^N$ - точка локального минимума (максимума). Пусть $f(X),\psi_j(X)$ - дважды непрерывно дифференцируемы и пусть
  в точке $Y\; J(Y)=\bigg\{\dfrac{\partial \psi_j(Y)}{\partial x_i}\bigg\}$ имеет ранг равный $M$.\\
  Тогда в стационарной точке функции Лагранжа $(Y)$ $ \forall\:\delta X\neq 0 \;\big| grad\:\psi_j(X)\cdot\delta X=0$ выполняется неравенство
  $\delta X L_{XX}^{''}(\Lambda^{'},Y)\delta X^{T}\geqslant (\leqslant) 0$.
\end{theorem}
\begin{theorem}[Достаточное условие экстремума]
  Пусть $Y\in D\subseteq \mathbb{R}^N$ - точка экстремума и $\psi_j(X)=0$. Пусть $f(X),\psi_j(X)$ - дважды непрерывно дифференцируемы. Если
  существуют $\Lambda^{'}=(\lambda_1,...,\lambda_M)\neq 0 \land \lambda_0^{'}\neq 0\;\big| grad\:L(\Lambda^{'},\lambda_0^{'},Y)=0$ и при этом
  $ \delta X L_{XX}^{''}(\Lambda^{'},Y)\delta X^{T}>(<) 0\; \forall\:\delta X\neq 0$ для которых $grad\:\psi_j(Y)\delta X =0$.\\
  Тогда $Y$ - точка локального минимума (максимума).
\end{theorem}
\begin{theorem}[Достаточное условие экстремума в терминах матрицы Гессе функции Лагранжа]
  Пусть найдена стационарная точка функции Лагранжа. \\
  $Y$ - точка максимума, если начиная с углового минора порядка $2M+1$ последующие $N-M$ угловых миноров матрицы Гессе образуют знакочередующийся
  числовой ряд в котором знак первого члена совпадает со знаком $(-1)^{M+1}$.\\
  $Y$ - точка минимума, если начиная с углового минора порядка $2M+1$ последующие $N-M$ угловых миноров матрицы Гессе имеют знак $(-1)^M$.
\end{theorem}

\section{Задача выпуклой оптимизации}
\begin{lemma}
  Пересечение конечного или счетного числа выпуклых множеств есть выпуклое множество.
\end{lemma}
\begin{lemma}
  Линейная комбинация $ \sum_{i=1}^N\alpha_iX_i$ конечного числа выпуклых множеств $X_i$ при любых $\alpha_i$ является выпуклым множеством.
\end{lemma}
\begin{lemma}
  Если $f_1(X),f_2(X),...,f_M(X)$ выпуклы (вогнуты) на выпуклом множестве $D$, то их линейная комбинация с неотрицательными коэффициентами $\displaystyle f(X)=\sum_{j=1}^M \alpha_jf_j(X) $ будет выпуклой (вогнутой) функцией на $D$.
\end{lemma}
\begin{lemma}
Пусть $O$ - выпуклое множество, $D$ - произвольное множество.\\Пусть $g(X,Y): O\in X\times D\in Y$. Пусть $g$ выпукла по $X$ на $O$ при $\forall\:Y$ и ограничена сверху по $Y$ при $\forall \:X$.\\Тогда $f(X)=\sup\limits_{Y\in D}g(X,Y)$ выпукла на $O$.
\end{lemma}
\begin{lemma}
  Если функции $g_1(X),g_2(X), ..., g_M(X)$ выпуклы на выпуклом множестве $O \subset \mathbb{R}^N$ и $G(X) = (g_1(X), g_2(X), ..., g_M(X))$ - вектор-функция, образованная из них, $q$ - монотонно неубывающая выпуклая функция, заданная на выпуклом множестве $D \subset \mathbb{R}^M$, и функция
$G(X)$ принимает значения из $D$, то функция $f (X) = q (G(X))$ выпукла на $O$.
\end{lemma}
\begin{lemma}
  Если функция $g$ выпукла на выпуклом множестве $O \subset \mathbb{R}^M$, $A$ - матрица размера $M\times N$, $B \in \mathbb{R}^M$ - вектор и множество $D=\bigg\{X \in\mathbb{R}^N: A\cdot X + B \in O\bigg\}$ непусто, то функция $f (X) = g(A\cdot X + B)$ выпукла на $D$.
\end{lemma}
\begin{lemma}[Дифференциальный критерий выпуклости]
  Дважды непрерывно дифференцируемая функция $f(X)$ выпукла (вогнута), если ее матрица Гессе является положительно полуопределенной (отрицательно полуопределенной). Если $H_f(Y)$ положительно (отрицательно) определена, то $f(X)$ строго выпукла (вогнута).
\end{lemma}

\textbf{Выпуклая задача оптимизации: }$(*)$
\begin{align}
   &f(Y)=extr_D\:f(X)\notag\\
   &D=\bigg\{ X\:\bigg|X\in P, \psi_j(X)\leqslant(\geqslant,=)0,j=1,...,M\bigg\}\subseteq \mathbb{R}^N\notag\\
   &D \text{ - выпуклое множество},f(x)\text{ - выпукла на }D\notag
\end{align}
\begin{theorem}[Условие выпуклости множества допустимых решений]
  Если $\psi(X)$ выпуклая (вогнутая) функция, то множество допустимых решений удовлетворяющее системе $\psi(X)\leqslant b, x_i\geqslant 0 \;(\psi(X)\geqslant b, x_i\geqslant 0)$ будет выпуклым.
\end{theorem}
\begin{theorem}[Необходимо условие экстремума]
Если в задаче $(*)$ целевая функция задана на выпуклой области определения и дифференцируема в $Y\in D$ и если $Y$ - точка локального минимума (максимума), то $grad\:f(Y)\cdot\delta X\geqslant(\leqslant)\:0$. $(\delta X=X-Y)$
\end{theorem}
\begin{lemma}
  Если точка локального экстремума $Y$ является внутренней точкой $D$, то $grad\:f(Y)=0.$
\end{lemma}
\begin{lemma}
  Пусть $D=\bigg\{X\;\big|X\in\mathbb{R}^N,a_j\leqslant x_j\leqslant b_j,j=1,...,N  \bigg\}$, $-\infty \leqslant a_j\leqslant b_j \leqslant \infty$.
  Тогда
  \begin{equation}
    \frac{\partial f(Y)}{\partial x_j}=\begin{cases}
      \geqslant (\leqslant) 0,\quad y_j=a_j\neq -\infty\\
      0,\quad\quad\quad\quad a_j\leqslant y_j\leqslant b_j\\
      \leqslant (\geqslant) 0,\quad y_j=b_j\neq +\infty
    \end{cases}
  \end{equation}
\end{lemma}
\begin{lemma}
  Пусть $D=\bigg\{X\;\big|X\in\mathbb{R}^N, x_j \geqslant 0,j=1,...,S  \bigg\}$.\\
  Тогда в точке локального минимума (максимума) \\
  при $j=1,...,S: \dfrac{\partial f(Y)}{\partial x_j}\geqslant(\leqslant)0$ если $y_j=0$ и $ \dfrac{\partial f(Y)}{\partial x_j}=0$ если $y_j>0$; \\
  при $j=S+1,...,S+N:$ $\dfrac{\partial f(Y)}{\partial x_j}=0 \;\forall\:y_j$.
\end{lemma}
\begin{theorem}[Достаточное условие экстремума]
Если в задаче $(*)$ целевая функция задана на выпуклой области определения и дифференцируема в $Y\in D$ и если $grad \:f(Y)\cdot\delta X\geqslant(\leqslant)\:0$, то  $Y$ точка $\min\: (\max)$.
\end{theorem}
\begin{lemma}
  Пусть $f (X)$ - выпуклая (вогнутая) функция, определенная на $D \subseteq \mathbb{R}^N$ и дифференцируемая во внутренней точке $Y \in D$.
Если $Y$ - стационарная точка функции $f (X), grad\: f(Y) = 0$, то $Y$ - точка экстремума $f(X)$ на $D$.
\end{lemma}
\begin{theorem}[Единственность точки экстремума задачи выпуклой оптимизации]
  Если выпуклая функция $f(X)$ определенная на $D$ имеет точку локального минимума (максимума), то эта точка является точкой глобального минимума (максимума).
\end{theorem}
\begin{theorem}
  Пусть $f(X)$ выпуклая функция определенная на $D$. Пусть $f(X)$ достигает глобального минимума (максимума) на $E$.\\
  Тогда $E$ выпуклое множество. $(E$ - множество точек глобального минимума (максимума) функции $f(X))$
\end{theorem}
\textbf{Общая (неклассическая) постановка задачи оптимизации: }$(**)$
\begin{align}
   &f(Y)=extr_D\:f(X)\notag\\
   &D=\bigg\{ X\:\bigg|X\in P; \psi_j(X)\leqslant 0,j=1,...,K;\psi_j(X)= 0,j=K+1,K+2,...,M \bigg\}\subseteq \mathbb{R}^N\notag\\
\end{align}
Функция Лагранжа: $\displaystyle L(\Lambda,\lambda_0,X)=\lambda_0f(X)+\sum_{j=1}^M\lambda_j\psi_j(X)$
\begin{theorem}[Необходимое условие экстремума в форме принципа Лагранжа]
  Пусть есть задача $(**)$. Пусть выполняются следующие условия:
  \begin{enumerate}
    \item $P$ - выпуклое множество;
    \item $f(X),\psi_1(X),...,\psi_K(X)$ дифференцируемы в точке $Y\in D$;
    \item $\psi_{K+1}(X),...,\psi_{K+M}(X)$ непрерывно дифференцируемы в окрестности $U_{\varepsilon}(Y)$.
  \end{enumerate}
Если $Y$ - точка локального минимума (максимума) задачи $(**)$ и при этом $\exists\;\Lambda^{'}=(\lambda_1^{'},...,\lambda_M^{'})\neq 0
\land \lambda_0^{'}\neq 0 \;\big|\forall\:X\in P$ и $\forall\: \delta X=X-Y$ выполняются условия Куна-Таккера.
\begin{align}
  &\sum_i \frac{\partial L(\Lambda^{'},\lambda_0^{'},Y)}{\partial x_i}\delta x_i \geqslant (\leqslant) 0\\
  &\lambda_j^{'}\psi_j(Y)=0, j=1,...,K\\
  &\lambda_j^{'}\geqslant (\leqslant) 0, j=1,...,K
\end{align}
$\lambda_{K+1}^{'},...,\lambda_M^{'}$ могут иметь любой знак.
\end{theorem}

\begin{theorem}[Достаточное условие экстремума]
  Пусть есть задача $(**)$. Пусть выполняются следующие условия:
  \begin{enumerate}
    \item $P$ - выпуклое множество;
    \item $f(X),\psi_1(X),...,\psi_K(X)$ дифференцируемы в точке $Y\in D$;
    \item $f(X),\psi_1(X),...,\psi_K(X)$ выпуклы на $P$;
    \item $\psi_{K+1}(X),...,\psi_{K+M}(X)$ линейны.
  \end{enumerate}
  Если существуют такие $\Lambda^{'}\neq 0 \land \lambda_0^{'}\neq 0$, что $\forall\:X\in P $ выполняются условия Куна-Таккера.\\
  Тогда $Y$ - точка минимума (максимума).
\end{theorem}

\begin{lemma}
  Пусть $Y$ - точка минимума (максимума). Пусть $Y$ - внутренняя точка $P$.\\
  Тогда $\dfrac{\partial L(\Lambda^{'},\lambda_0^{'},Y)}{\partial x_i}=0, i=1,...,N$.\\
  Если $P=\bigg\{X\in P\;\bigg| a_i\leqslant x_i \leqslant b_i, i=1,...,N \bigg\}$.\\
  Тогда
  \begin{equation}
    \frac{\partial L}{\partial x_i}=\begin{cases}
      \geqslant (\leqslant) 0,\quad y_i=a_i\neq -\infty\\
      0,\quad\quad\quad\quad a_i< y_j< b_i\\
      \leqslant (\geqslant) 0,\quad y_i=b_i\neq +\infty
    \end{cases}
  \end{equation}
  Если $P=\bigg\{X\in P\;\bigg| x_i \geqslant 0, i=1,...,S, 0\leqslant S \leqslant N \bigg\}$.\\
  Тогда
  \begin{align}
    &\frac{\partial L(\Lambda^{'},\lambda_0^{'},Y)}{\partial x_i}\geqslant (\leqslant) 0; y_i\frac{\partial L(\Lambda^{'},\lambda_0^{'},Y)}{\partial x_i}=0,\qquad i=1,...,S\\
    &\frac{\partial L(\Lambda^{'},\lambda_0^{'},Y)}{\partial x_i}=0,\qquad\;\; i=S+1,S+2,...,N
  \end{align}
\end{lemma}
\textbf{Условие регулярности:} линейная независимость набора градиентов ограничений в $D$.
\begin{theorem}[Необходимое условие экстремума Куна-Таккера]
  Пусть есть задача $(**)$. Пусть выполняются следующие условия:
  \begin{enumerate}
    \item $P$ - выпуклое множество;
    \item $f(X),\psi_1(X),...,\psi_K(X)$ дифференцируемы в точке $Y\in D$;
    \item $\psi_1(X),...,\psi_K(X)$ выпуклы на $P$;
    \item $\psi_{K+1}(X),...,\psi_{K+M}(X)$ линейны.
  \end{enumerate}
  И выполнено одно из условий:
  \begin{enumerate}
    \item[a)] ограничения равенства отсутствуют, т.е. $K=M$ и система $\psi_j(X)<0,j=1,...,M$ имеет решение на $P$;
    \item[b)] $P$ - полиэдр и $\psi_j(X),j=1,...,K$ - линейны;
    \item[c)] $P$ - полиэдр и $\psi_{S+1}(X),...,\psi_K$ - линейны и система ограничений $\psi_j(X)\leqslant 0,j=1,...,S$ имеет хотя бы одно допустимое решение.
  \end{enumerate}
  Если $Y$ - точка локального минимума (максимума)\\
  Тогда существуют такие $\Lambda^{'}\neq 0 \land \lambda_0^{'}\neq 0$, что $\forall\:X\in P $ выполняются условия Куна-Таккера.
\end{theorem}

\begin{theorem}[Необходимые и достаточные условия экстремума Куна-Таккера в дифф. форме]
  Пусть есть задача $(**)$. Пусть выполняются следующие условия:
  \begin{enumerate}
    \item $P$ - выпуклое множество;
    \item $f(X),\psi_1(X),...,\psi_K(X)$ дифференцируемы в точке $Y\in D$;
    \item $\psi_1(X),...,\psi_K(X)$ выпуклы на $P$;
    \item $\psi_{K+1}(X),...,\psi_{K+M}(X)$ линейны;
    \item $f(x)$ - выпукла.
  \end{enumerate}
  И выполняется одно из $a)-c)$.\\
  Точка $Y$ локального минимума (максимума) существует $\iff$ $\exists\: \Lambda^{'}\neq 0 \land \lambda_0^{'}\neq 0$, такие что $\forall\:X\in P $ выполняются условия Куна-Таккера.
\end{theorem}

\textbf{Задача: }$(***)$
\begin{align}
  &f(Y)=\min{f(X)}\notag\\
  &D=\bigg\{ X\:\bigg|X\in P, \psi_j(X)\leqslant 0,j=1,...,K;\psi_j(X)= 0,j=K+1,...,M\bigg\}\subseteq \mathbb{R}^N\notag
\end{align}

\begin{theorem}
  Пусть есть задача $(***)$. Пусть выполняются следующие условия:
  \begin{enumerate}
    \item $P$ - выпуклое множество;
    \item $f(X),\psi_1(X),...,\psi_K(X)$ выпуклы на $P$;
    \item $\psi_{K+1}(X),...,\psi_{K+M}(X)$ линейны;
    \item $D$ - непусто.
  \end{enumerate}
  Тогда существуют такие $\Lambda^{'}\neq 0 \land \lambda_0^{'}\neq 0$, что $\forall\:X\in P $ выполняются неравенства
  \begin{align}
    &\lambda_0^{'}f^{*}\leqslant \lambda_0^{'}f(X)+\sum_{j=1}^M \lambda_j \psi_j(X)=L(\Lambda^{'},\lambda_0^{'},X)\\
    &\lambda_j^{'}\geqslant 0, j=1,...,K
  \end{align}
\end{theorem}

\begin{theorem}[Достаточное условие существования вектора Куна-Таккера]
  Пусть есть задача $(***)$. Пусть выполняются следующие условия:
  \begin{enumerate}
    \item $P$ - выпуклое множество;
    \item $f(X),\psi_1(X),...,\psi_K(X)$ выпуклы на $P$;
    \item $\psi_{K+1}(X),...,\psi_{K+M}(X)$ линейны.
  \end{enumerate}
  И выполняется одно из $a)-c)$. Тогда вектор Куна-Таккера существует.
\end{theorem}
\textbf{Условие регулярности задачи $(***)$: }существование вектора Куна-Таккера.

\begin{theorem}[Выпуклость двойственной задачи]
  В двойственной задаче $Q$ - выпукло и $\varphi$ вогнута (выпукла вверх) на $Q$.
\end{theorem}

\begin{theorem}[Основное неравенство двойственности для задачи выпуклой оптимизации]
  $\forall\:X\in D$ прямой задачи и $\forall\:\Lambda\in \mathcal{L}$ двойственной задачи справедливо неравенство $f(X)\geqslant \varphi(\Lambda)$.
\end{theorem}

\begin{theorem}[Теорема двойственности]
  Если прямая задача имеет решение и оно конечно и выполнено условие регулярности (th 11).\\
  Тогда множество решений двойственной задачи непусто и совпадает со множеством векторов Куна-Таккера прямой задачи. И целевые функции прямой и двойственной задач совпадают.
\end{theorem}

\begin{theorem}[Связь между решением прямой и двойственной задачи]
  Если для прямой задачи $(***)$ выполнено условие регулярности и допустимое множество двойственной задачи непусто, то двойственная задача имеет решение.\\ Если допустимое множество пусто, то минимум прямой задачи это <<$-\infty $>>.
\end{theorem}

\begin{theorem}[Теорема Куна-Таккера в форме двойственности]
  Если выполнено условие теоремы 11 для прямой задачи, то точка $Y$ есть решение прямой задачи $\iff$ существует вектор Куна-Таккера, такой что $f(Y)=\varphi(\Lambda)$.
\end{theorem}

\begin{theorem}[Теорема Куна-Таккера в терминах седловой точки]
  Если выполнено условие теоремы 11 для прямой задачи, то точка $Y$ есть решение прямой задач $\iff$ существует вектор Куна-Таккера, такой что $(Y,\Lambda^{'})$ - седловая точка функции Лагранжа.
\end{theorem}

\begin{theorem}[Об условиях одновременного достижения экстремума прямой и двойственной задачи]
  Если выполнено условие теоремы 11, то точки $Y$ и $\Lambda^{'}$ есть решения прямой и двойственной задач $\iff$ выполнено соотношение двойственности.\\ Или точки $Y$ и $\Lambda^{'}$ есть решения прямой и двойственной задач $\iff$ $(Y,\Lambda^{'})$ - седловая точка функции Лагранжа.
\end{theorem}
\section{Задача выпуклой квадратичной оптимизации}

\begin{theorem}
  Для того, чтобы существовал вектор Куна-Таккера $\Lambda^{'}\in \mathcal{L}$, удовлетворяющий условиям $\begin{cases}
    CX+S+\Lambda^{'}=0\\
    \lambda_j^{'}(A_jY-b_j)=0, j=1,...,K
\end{cases}$ $\iff$ $Y\in D$ - точкой минимума.
\end{theorem}
\end{document}
