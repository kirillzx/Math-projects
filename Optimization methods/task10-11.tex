\documentclass[12pt]{article}
 \usepackage[margin=1in]{geometry}
\usepackage{amsmath,amsthm,amssymb,amsfonts, enumitem, fancyhdr, color, comment, graphicx, environ, fancyhdr}
\pagestyle{fancy}
\usepackage{cmap}
\usepackage[T2A]{fontenc}
\usepackage[utf8]{inputenc}
\usepackage[english,russian]{babel}
\usepackage[table]{xcolor}

\newcolumntype{s}{>{\columncolor[HTML]{C05DFC}} c}
% \cellcolor[HTML]{FFBF00}

\setlength{\parindent}{0ex}
\setlength{\headheight}{35pt}




\lhead{Optimization theory}
\rhead{Kirill Zakharov}

\begin{document}
\section*{Task 10 (Метод ветвей и границ)}
Целевая функция и ограничения имеют следующий вид
\begin{align}
  &f = 10x_1-6x_2\to\max\notag\\
  &x_1-8x_2\leq10\notag\\
  &x_1+x_2\geq6\\
  &x_1\geq2\notag\\
  &x_2\leq4.99\notag\\
  &x_1,x_2\geq0\notag
\end{align}
Решение данной задачи: $x^*=(49.92;4.99), f^*=469.26$ - не целые, поэтому применим метод ветвей и границ.
Начнем с переменной $x_1$.
\begin{enumerate}
  \item $x_1\leqslant [x_1^*]=49$ и $x_1\geqslant [x_1^*]+1=50$
    \begin{enumerate}
      \item[1.1)] $f = 10x_1-6x_2\to\max; \;\text{constraints};\; x_1\leqslant 49$\\
      $x_{1.1}^*=(49;4.875), f^*=460.75$\\
      \begin{enumerate}
        \item[1.1.1)]  $f = 10x_1-6x_2\to\max;\; \text{constraints};\; x_1\leqslant 49;\; x_2\leqslant 4$\\
        $x_{1.1.1}^*=(42;4), f^*=396$\\
        \item[1.1.2)]  $f = 10x_1-6x_2\to\max;\; \text{constraints};\; x_1\leqslant 49;\; x_2\geqslant 5$\\
        $x_{1.1.2}^*\in\varnothing$\\
      \end{enumerate}
      \item[1.2)] $f = 10x_1-6x_2\to\max; \;\text{constraints};\; x_1\geqslant 50$\\
      $x_{1.2}^*\in\varnothing$\\

    \end{enumerate}
  \item $x_2\leqslant [x_2^*]=4$ и $x_2\geqslant [x_2^*]+1=5$
  \begin{enumerate}
    \item[2.1)] $f = 10x_1-6x_2\to\max; \;\text{constraints};\; x_2\leqslant 4$\\
    $x_{2.1}^*=(42;4), f^*=396$\\
    \item[2.2)] $f = 10x_1-6x_2\to\max; \;\text{constraints};\; x_2\geqslant 5$\\
    $x_{2.2}^*\in\varnothing$
  \end{enumerate}
\end{enumerate}
Таким образом выбираем решение $x=(42; 4), f=396$.

\section*{Task 11 (Метод Гомори)}
Целевая функция и ограничения имеют следующий вид
\begin{align}
  &f = 10x_1-6x_2\to\max\notag\\
  &x_1-8x_2\leq10\notag\\
  &x_1+x_2\geq6\\
  &x_1\geq2\notag\\
  &x_2\leq4.99\notag\\
  &x_1,x_2\geq0\notag
\end{align}
Приведем задачу к каноническому виду, добавив слабые и искусственные переменные.
\begin{align}
  &x_1-8x_2+y_1=10\notag\\
  &x_1+x_2-y_2+R_1=6\notag\\
  &x_1-y_3+R_2=2\notag\\
  &x_2+y_4=4.99\\
  &x_{1,2}\geqslant 0;\: y_j\geqslant 0\notag\\
  &R_{1,2}\geqslant 0\notag
\end{align}
Найдем оптимальное решение при помощи двухэтапного метода.\\
\begin{center}
\includegraphics[scale=0.67]{images11/i1.png}
\includegraphics[scale=0.67]{images11/i2.png}
\includegraphics[scale=0.7]{images11/i3.png}
\includegraphics[scale=0.7]{images11/i4.png}
\end{center}
Получили следующее оптимальное решение: $x_1=49.92, x_2=4.99, f= 469.26$. Решение не является целочисленным, поэтому
применим метод Гомори. Из таблицы видно, что $x_2+y_4=4.99\Rightarrow x_2=4.99-y_4\Rightarrow 0.99-y_4\leqslant 0
\Rightarrow 99-100y_4\leqslant 0\Rightarrow-100y_4+s_1=-99$. Добавим новое ограничение в оптимальное решение и применим двойственный симплекс-метод.
\center{\includegraphics[scale=0.7]{images11/i5.png}}
\end{document}
