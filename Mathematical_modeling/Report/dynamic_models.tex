% \documentclass[12pt]{article}
%  \usepackage[margin=1in]{geometry}
% \usepackage{amsmath, amsthm, amssymb, amsfonts, enumitem, fancyhdr, color, comment, graphicx, environ, fancyhdr}
% \pagestyle{fancy}
% \usepackage{cmap}
% \usepackage[T2A]{fontenc}
% \usepackage[utf8]{inputenc}
% \usepackage[english, russian]{babel}
%
% \newtheorem*{exersize}{Упражнение}
% \newtheorem*{lemma}{Лемма}
% \newtheorem*{theorem}{Теорема}
% \newtheorem*{consequence}{Следствие}
% \newtheorem*{statement}{Утверждение}
% \newtheorem{property}{Свойство}
% \newtheorem*{fact}{Факт}
%
% \theoremstyle{definition}
% \newtheorem*{definition}{Определение}
% \newtheorem*{problem}{Задача}
% \newtheorem*{example}{Пример}
%
% \theoremstyle{remark}
% \newtheorem*{note}{Замечание}
% \setlength{\parindent}{0pt}
% \setlength{\headheight}{35pt}
%
%
% \begin{document}

\section{Динамические модели}
\subsection{Модель Солоу}
Пусть имеется замкнутая односекторная экономика.\\
$Y$ - ВВП\\
$K$ - капитал\\
$I$ - инвестиции\\
$C$ - конечное потребление\\
$L$ - трудовые ресурсы\\
Имеется баланс $Y=C+I$. Зависимость ВВП от ресурсов выражается функцией Кобба-Дугласа.
\begin{align}
  &Y=AK^{\alpha}L^{\beta}\notag\\
  &Y=C+I\notag\\
  &I=sY\\
  &\dfrac{\partial L}{\partial t}=\gamma L \qquad \qquad\quad\,(L(0)=L_0)\notag\\
  &\dfrac{\partial K}{\partial t}=-\mu K+I \qquad(K(0)=K_0)\notag
\end{align}
где $ \gamma$ - темп прироста трудовых ресурсов, $s$ - склонность к сбережению, $A$ - научно-технический прогресс.
Пусть $y=Y/L,k=K/L,i=I/L$. Тогда получим модель Солоу в относительных показателях.
\begin{equation}
  \frac{\partial k}{\partial t}=-(\lambda+\mu)k+sAk^{\alpha}
\end{equation}
Равновесие равно $\hat{k}=(\dfrac{sA}{\lambda+\mu})^{\frac{1}{1-\alpha}}$
\begin{center}
\begin{tabular}{|c|c|}
  \hline
  Интервалы&Рост\\
  \hline
  $\bigg(0;(\dfrac{\alpha sA}{\lambda+\mu})^{\frac{1}{1-\alpha}}\bigg)$&Ускоренный рост\\
  \hline
  $\bigg((\dfrac{\alpha sA}{\lambda+\mu})^{\frac{1}{1-\alpha}};(\dfrac{sA}{\lambda+\mu})^{\frac{1}{1-\alpha}}\bigg)$&Насыщенный рост\\
  \hline
  $\bigg((\dfrac{sA}{\lambda+\mu})^{\frac{1}{1-\alpha}};+\infty\bigg)$&Падение\\
  \hline
\end{tabular}
\end{center}
\textbf{Конечно-разностное представление}:
$k(t+\Delta)=k(t)+\Delta t(-(\lambda+\mu)k(t)+sAk(t)^{\alpha})$

\subsection{SIR модель}
Пусть $S(t)$ - число восприимчивых к инфекции\\
$I(t)$ - число инфицированных\\
$R(t)$ - число переболевших инфекцией\\
$N$ - число популяции\\
$\beta$ - коэффициент интенсивности контактов \\
$\gamma$ - коэффициент интенсивности выздоровления
\begin{align}
  &\dfrac{d S}{dt}=\dfrac{-\beta IS}{N}\notag\\
  &\dfrac{d I}{dt}=\dfrac{\beta IS}{N}-\gamma I\\
  &\dfrac{d R}{dt}=\gamma I\notag
\end{align}

\subsection{SEIRD модель}
$E(t)$ - число носителей заболевания\\
$D$ - число умерших\\
$\mu$ - уровень смертности\\
$\delta=\dfrac{1}{\text{ср.инк.период}}$
\begin{align}
  &\dfrac{d S}{dt}=\dfrac{-\beta IS}{N}\notag\\
  &\dfrac{d E}{dt}=\dfrac{\beta IS}{N}-\delta E\notag\\
  &\dfrac{d I}{dt}=\delta E-\gamma I-\mu I\\
  &\dfrac{d R}{dt}=\gamma I\notag\\
  &\dfrac{d D}{dt}=\mu I\notag
\end{align}

\subsection{Модель Лотки-Вольтерра}
\begin{equation}
  \begin{cases}
    \dot{x}=ax-bxy\\
    \dot{y}=-cy+dxy
  \end{cases}
\end{equation}
$x(t)$ - число жертв\\
$y(t)$ - число хищников\\
$a$ - коэффициент рождаемости жертв\\
$b$ - коэффициент убыли жертв\\
$c$ - коэффициент убыли хищников \\
$d$ - коэффициент рождаемости хищников\\
Первой стационарной точкой является $(0,0)$. Возьмем из системы линейную часть и составим матрицу.
\begin{equation}
  \begin{vmatrix}
  a-\lambda&0\\0&-c-\lambda\;
\end{vmatrix}
\end{equation}
Решая данной характеристическое уравнение получим $\lambda_1=a, \lambda_2=-c\Rightarrow$ данная точка является седлом.
Теперь приравняем правые части системы к 0 и решим ее. Получим вторую стационарную точку $\overline{x}=\dfrac{c}{d},
\overline{y}=\dfrac{a}{b}$. Построим матрицу Якоби, подставив $\overline{x},\overline{y}$.
\begin{equation}
  \begin{pmatrix}
    0&-\dfrac{bc}{d}\\-\dfrac{ad}{b}&0
  \end{pmatrix}
\end{equation}
Решая характеристическое уравнение $\lambda^2+ac=0$, получаем два мнимых корня, что свидетельствует о том что данная
стационарная точка является центром.

\subsection{Модель взаимодействия двух конкурирующих видов}
$x_1$ - количество особей первого типа
$x_2$ - количество особей второго типа
\begin{equation}
  \begin{cases}
    \dot{x}_1=a_1x_1-b_{11}x_1^2-b_{12}x_1x_2\\
    \dot{x}_2=a_2x_2-b_{21}x_1x_2-b_{22}x_2^2
  \end{cases}
\end{equation}
Приравняем к 0 правые части системы.
\begin{equation}
  \begin{cases}
    x_1(a_1-b_{11}x_1-b_{12}x_2)=0\\
    x_2(a_2-b_{21}x_1-b_{22}x_2)=0
  \end{cases}
\end{equation}
Получим 4 стационарные точки.
\begin{gather}
  \begin{cases}
    x_1=0\\
    x_2=0
  \end{cases};
  \begin{cases}
    x_1=0\\
    x_2=\dfrac{a_2}{b_{22}}
  \end{cases};\\
  \begin{cases}
    x_1=\dfrac{a_1}{b_{11}}\\
    x_2=0
  \end{cases};
  \begin{cases}
    x_1=\dfrac{a_2b_{12}-a_1b_{22}}{b_{12}b_{21}-b_{22}b_{11}}\\
    x_2=\dfrac{a_1b_{21}-a_2b_{11}}{b_{12}b_{21}-b_{22}b_{11}}
  \end{cases}\notag
\end{gather}
% \end{document}
