% \documentclass[12pt]{article}
%  \usepackage[margin=1in]{geometry}
% \usepackage{amsmath,amsthm,amssymb,amsfonts, enumitem, fancyhdr, color, comment, graphicx, environ, fancyhdr}
% \pagestyle{fancy}
% \usepackage{cmap}
% \usepackage[T2A]{fontenc}
% \usepackage[utf8]{inputenc}
% \usepackage[english,russian]{babel}
% \setlength{\parindent}{5ex}
% \setlength{\headheight}{35pt}
%
% \lhead{Mathematical modeling}
% \rhead{Kirill Zakharov}
%
% \begin{document}
\subsection{Переход к полярным координатам}
\begin{equation}
\begin{cases}
  \dot{x}_1=x_1-x_2-x_1(x_1^2+x_2^2)\\
  \dot{x}_2=x_1+x_2-x_2(x_1^2+x_2^2)
\end{cases}
\end{equation}
Перейдем к полярным координатам.
\begin{equation}
  \begin{cases}
    x_1(t)=r(t)\cos{\varphi(t)}\\
    x_2(t)=r(t)\sin{\varphi(t)}
  \end{cases}
\end{equation}
Выполним подстановку и получим выражения для $\dot{r}$ и $\dot{\varphi}$.
\begin{align}
  &\begin{cases}
    \dot{r}\cos{\varphi}+r\dot{\varphi}(-\sin{\varphi})=r\cos{\varphi}-r\sin{\varphi}-r^3\cos{\varphi}\;\big|*\cos{\varphi}\\
    \dot{r}\sin{\varphi}+r\dot{\varphi}\cos{\varphi}=r\cos{\varphi}+r\sin{\varphi}-r^3\sin{\varphi}\quad\quad\big|*\sin{\varphi}
  \end{cases}+\\
  &\begin{cases}
    \dot{r}\cos^2{\varphi}-r\dot{\varphi}\sin{\varphi}\cos{\varphi}=
    r\cos^2{\varphi}-r\cos{\varphi}\sin{\varphi}-r^3\cos^2{\varphi}\\
    \dot{r}\sin^2{\varphi}+r\dot{\varphi}\cos{\varphi}\sin{\varphi}=
    r\sin{\varphi}\cos{\varphi}+r\sin^2{\varphi}-r^3\sin^2{\varphi}
  \end{cases}
\end{align}
Тем самым получаем выражение для $\dot{r}(t)=r(t)(1-r^2(t))$. Теперь умножим первое уравнения на $sin{\varphi}$,
второе на $cos{\varphi}$ и вычтем из первого второе.
\begin{align}
  &\begin{cases}
    \dot{r}\cos{\varphi}+r\dot{\varphi}(-\sin{\varphi})=r\cos{\varphi}-r\sin{\varphi}-r^3\cos{\varphi}\;\big|*\sin{\varphi}\\
    \dot{r}\sin{\varphi}+r\dot{\varphi}\cos{\varphi}=r\cos{\varphi}+r\sin{\varphi}-r^3\sin{\varphi}\quad\quad\big|*\cos{\varphi}
  \end{cases}-\\
  &\begin{cases}
    \dot{r}\cos{\varphi}\sin{\varphi}-r\dot{\varphi}\sin^2{\varphi}=
    r\cos{\varphi}\sin{\varphi}-r\sin^2{\varphi}-r^3\cos{\varphi}\sin{\varphi}\\
    \dot{r}\sin{\varphi}\cos{\varphi}+r\dot{\varphi}\cos^2{\varphi}=
    r\cos^2{\varphi}+r\sin{\varphi}\cos{\varphi}-r^3\sin{\varphi}\cos{\varphi}
  \end{cases}
\end{align}
Таким образом получим систему (1) в полярных координатах.
\begin{equation}
  \begin{cases}
    \dot{r}=r(1-r^2)\\
    \dot{\varphi}=1
  \end{cases}
\end{equation}
Стационарные точки для данной системы $r=0$ и $r=1$.\\
Устойчивый вариант:
\begin{equation}
  \begin{cases}
    \dot{r}=r(1-r^2)\\
    \dot{\varphi}=1
  \end{cases}
\end{equation}
Неустойчивый вариант:
\begin{equation}
  \begin{cases}
    \dot{r}=r(r^2-1)\\
    \dot{\varphi}=1
  \end{cases}
\end{equation}
Полуустойчивый вариант:
\begin{equation}
  \begin{cases}
    \dot{r}=r(1-r)^2\\
    \dot{\varphi}=1
  \end{cases}
\end{equation}
% \end{document}
