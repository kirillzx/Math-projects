\documentclass[12pt]{article}
 \usepackage[margin=1in]{geometry}
\usepackage{amsmath, amsthm, amssymb, amsfonts, enumitem, fancyhdr, color, comment, graphicx, environ, fancyhdr}
\pagestyle{fancy}
\usepackage{cmap}
\usepackage[T2A]{fontenc}
\usepackage[utf8]{inputenc}
\usepackage[english, russian]{babel}
\usepackage{caption}
\usepackage{indentfirst}
% \usepackage{hyperref}

% \usepackage{color} \definecolor{darkgreen}{rgb}{0,0,0}
% \usepackage[unicode,colorlinks,filecolor=blue,citecolor=darkgreen,pagebackref]{hyperref}

\newtheorem*{exersize}{Упражнение}
\newtheorem*{lemma}{Lemma}
\newtheorem{theorem}{Теорема}[section]
\newtheorem*{consequence}{Следствие}
\newtheorem*{statement}{Statement}
\newtheorem{property}{Property}
\newtheorem*{fact}{Fact}

\theoremstyle{definition}
\newtheorem{definition}{Определение}[section]
\newtheorem*{problem}{Problem}
\newtheorem*{example}{Example}

\theoremstyle{remark}
\newtheorem*{remark}{Remark}
\setlength{\parindent}{5ex}
\setlength{\headheight}{35pt}
%
% \lhead{Mathematical modeling}
% \rhead{Kirill Zakharov}

\title{Математическое моделирование}
\author{Лектор: Лебедева Л.Н.\\Студент: Захаров К.А.}
\date{2020 г.}

\begin{document}
\maketitle
Все реализации моделей на моем github: \textit{https://github.com/kirillzx/Math-projects}
\tableofcontents

\section{Введение}
\textbf{Модель} - образ или прообраз какого-либо объекта или системы объектов, используемый в качестве их "заместителя".

\textbf{Математическая модель} - описание объекта исследования на языке математики.\\
Требования к модели:
\begin{itemize}
  \item адекватность
  \item конечность
  \item полнота(информированность)
  \item упрощенность
  \item гибкость
  \item приемлемая трудоемкость разработки
\end{itemize}
Этапы построения модели:
\begin{enumerate}
  \item определение цели;
  \item изучение предметной области, выявление причинно-следственных связей;
  \item переход от концептуальной модели к формализованному описанию;
  \item проверка адекватности моделирование;
  \item корректировка модели;
  \item применение модели. Проведение исследования и практическое использование.
\end{enumerate}
Классификация моделей:
\begin{itemize}
  \item линейные или нелинейные;
  \item сосредоточенные и распределенные системы;
  \item детерминированные или стохастические;
  \item статические или динамические;
  \item дискретные или непрерывные.
\end{itemize}

% \documentclass[12pt]{article}
%  \usepackage[margin=1in]{geometry}
% \usepackage{amsmath, amsthm, amssymb, amsfonts, enumitem, fancyhdr, color, comment, graphicx, environ, fancyhdr}
% \pagestyle{fancy}
% \usepackage{cmap}
% \usepackage[T2A]{fontenc}
% \usepackage[utf8]{inputenc}
% \usepackage[english, russian]{babel}
%
% \newtheorem*{exersize}{Упражнение}
% \newtheorem{lemma}{Lemma}[section]
% \newtheorem{theorem}{Theorem}[section]
% \newtheorem*{consequence}{Следствие}
% \newtheorem*{statement}{Statement}
% \newtheorem*{property}{Property}
% \newtheorem*{fact}{Fact}
%
% \theoremstyle{definition}
% \newtheorem{definition}{Definition}[section]
% \newtheorem*{problem}{Задача}
% \newtheorem*{example}{Пример}
%
% \theoremstyle{remark}
% \newtheorem*{note}{Замечание}
% \setlength{\parindent}{0pt}
% \setlength{\headheight}{35pt}
%
% \begin{document}
\section{Статические модели}
\subsection{Производственная функция Кобба-Дугласа}

\begin{definition}
  Производственная функция - функция выражающая зависимость между затратами ресурсов и объемом выпуска.
\end{definition}
Пусть $\overline{X}$ - вектор используемых ресурсов, $\overline{Y}$ - объем выпуска продукции каждого вида.
\begin{property}{О производственной функции}
\begin{enumerate}
  \item $F(x_1,...,x_n)$ является достаточно гладкой, т.е. $F\in C^2$
  \item $F(x_1,...,x_n)$ - возрастающая по каждому аргументу $\dfrac{\partial F}{\partial x_i}>0 \forall \;i$
  \item выпуск по каждому аргументу не ограничен
  \item предельная производительность убывает $ \dfrac{\partial^2 F}{\partial x_i^2}>0 \forall \;i$
\end{enumerate}
\end{property}

\begin{definition}{Однородная функция}
  \\$F(\lambda x_1, ...,\lambda x_n)=\lambda F(x_1,...,x_n)$
\end{definition}

Пусть $Y$ - это ВВП, $K$ - основные производственные фонды, $L$ – число занятых.
 Тогда определим функцию $Y=AK^{\alpha} L^{\beta},(A>0,0<\alpha,\beta<1)$.
Для оценки параметров $A,\alpha,\beta$ воспользуемся методом наименьших квадратов $\sum_{i=1}^n(a+b x_i-y_i )^2\to\min$. Также необходимо линеаризовать данные параметры при помощи натурального алгоритма. После чего получим следующую целевую функцию.
\begin{equation}
  S(A,\alpha,\beta)=\sum_{i=1}^M (\ln{A}+\alpha\ln{K_i}+\beta\ln{L_i}-\ln{Y_i})^2 \to\min
\end{equation}
Теперь нужно приравнять частные производные к нулю по каждому аргументу и решить систему линейных уравнений (3).
\begin{gather}
  \begin{cases}
  \dfrac{\partial S}{\partial \ln{A}}=0\\\\
  \dfrac{\partial S}{\partial \alpha}=0\\\\
  \dfrac{\partial S}{\partial \beta}=0
\end{cases};\\
\begin{cases}
  M\ln{A}+\alpha \sum_{i=1}^M\ln{K_i}+\beta \sum_{i=1}^M\ln{L_i}=\sum_{i=1}^M\ln{Y_i}\\\\
  ln{A}\sum_{i=1}^M\ln{K_i}+\alpha \sum_{i=1}^M\ln{K_i}^2+\beta\sum_{i=1}^M\ln{K_i}\ln{L_i}=\sum_{i=1}^M\ln{Y_i}\ln{K_i}\\\\
  ln{A}\sum_{i=1}^M\ln{L_i}+\alpha \sum_{i=1}^M\ln{K_i}\ln{L_i}+\beta\sum_{i=1}^M\ln{L_i}^2=\sum_{i=1}^M\ln{Y_i}\ln{L_i}
  \end{cases}
\end{gather}
Для решения данной модели воспользуемся средствами Excel.
\begin{center}
  \begin{tabular}{c}
    \includegraphics[scale=0.45]{images/cd.png}\\
    Рисунок 1 - Модель Кобба-Дугласа
  \end{tabular}
\end{center}
По данным ВВП США получено решение для коэффициентов $\alpha, \beta, A$.


\subsection{Модель Леонтьева (Межотраслевой баланс)}
Пусть $x_{ij}$ - промежуточный продукт, т.е. отрасль $i$ изготавливает продукт для отрасли $j$\\
$X_i$ - валовый продукт отрасли $i$\\
$Y_i$ - конечный продукт отрасли $i (i=\overline{1,n})$.\\
Тогда валовый выпуск определяется по следующей формуле.
\begin{equation}
X_i=\sum_{j=1}^nx_{ij}+Y_i
\end{equation}
Матрица прямых затрат определяется, как $a_{ij}=x_{ij}/X_j$ . Тогда вектор валового выпуска можно определить,
 как $X=AX+Y\Rightarrow X=(E-A)^{-1} Y$, а конечный продукт $Y=(E-A)X$.

 \begin{definition}
   Если хотя бы для одного положительного $Y$ уравнение межотраслевого баланса имеет неотрицательное решение, то матрица $A$ продуктивна.
 \end{definition}
 \begin{definition}
  Матрица $A$ продуктивна $\iff$ $(E-A)$ имеет $n$ положительных последовательностей главных миноров.
 \end{definition}
 \begin{definition}
  Матрица $A$ продуктивна $\iff$ когда матричный ряд $E+A+A^2+...+A_k+...$ сходится.
 \end{definition}
 \begin{definition}
  Матрицей полных затрат называется обратная матрица Леонтьева $B=(E-A)^{-1}$
 \end{definition}
На рисунках 2-3 представлено решение индивидуального задания по модели Леонтьева. В первом задание вычислены объемы конечно продукта при увеличение валового выпуска каждой отрасли на $10\%,50\%,20\%$ соответственно.
 \begin{center}
   \begin{tabular}{c}
     \includegraphics[scale=0.45]{images/ib_1.png}\\
     Рисунок 2 - Индивидуальное задание 1\\
     \includegraphics[scale=0.45]{images/ib_2.png}\\
     Рисунок 3 - Индивидуальное задание 2
   \end{tabular}
 \end{center}
Во втором задании вычислены валовые выпуски отраслей по заданной матрице прямых затрат и вектору конечной продукции.
На рисунках 4-5 показано решение на Python.
\begin{center}
  \begin{tabular}{c}
    \includegraphics[scale=0.45]{images/ib_3.png}\\
    Рисунок 4 - Модель Леонтьева\\
    \includegraphics[scale=0.45]{images/ib_4.png}\\
    Рисунок 5 - Индивидуальное задание
  \end{tabular}
\end{center}
% \end{document}

% \documentclass[12pt]{article}
%  \usepackage[margin=1in]{geometry}
% \usepackage{amsmath, amsthm, amssymb, amsfonts, enumitem, fancyhdr, color, comment, graphicx, environ, fancyhdr}
% \pagestyle{fancy}
% \usepackage{cmap}
% \usepackage[T2A]{fontenc}
% \usepackage[utf8]{inputenc}
% \usepackage[english, russian]{babel}
%
% \newtheorem*{exersize}{Упражнение}
% \newtheorem*{lemma}{Лемма}
% \newtheorem*{theorem}{Теорема}
% \newtheorem*{consequence}{Следствие}
% \newtheorem*{statement}{Утверждение}
% \newtheorem{property}{Свойство}
% \newtheorem*{fact}{Факт}
%
% \theoremstyle{definition}
% \newtheorem*{definition}{Определение}
% \newtheorem*{problem}{Задача}
% \newtheorem*{example}{Пример}
%
% \theoremstyle{remark}
% \newtheorem*{note}{Замечание}
% \setlength{\parindent}{0pt}
% \setlength{\headheight}{35pt}
%
%
% \begin{document}

\section{Динамические модели}
\subsection{Модель Солоу}
Пусть имеется замкнутая односекторная экономика.\\
$Y$ - ВВП\\
$K$ - капитал\\
$I$ - инвестиции\\
$C$ - конечное потребление\\
$L$ - трудовые ресурсы\\
Имеется баланс $Y=C+I$. Зависимость ВВП от ресурсов выражается функцией Кобба-Дугласа.
\begin{align}
  &Y=AK^{\alpha}L^{\beta}\notag\\
  &Y=C+I\notag\\
  &I=sY\\
  &\dfrac{\partial L}{\partial t}=\gamma L \qquad \qquad\quad\,(L(0)=L_0)\notag\\
  &\dfrac{\partial K}{\partial t}=-\mu K+I \qquad(K(0)=K_0)\notag
\end{align}
где $ \gamma$ - темп прироста трудовых ресурсов, $s$ - склонность к сбережению, $A$ - научно-технический прогресс.
Пусть $y=Y/L,k=K/L,i=I/L$. Тогда получим модель Солоу в относительных показателях.
\begin{equation}
  \frac{\partial k}{\partial t}=-(\lambda+\mu)k+sAk^{\alpha}
\end{equation}
Равновесие равно $\hat{k}=(\dfrac{sA}{\lambda+\mu})^{\frac{1}{1-\alpha}}$
\begin{center}
\begin{tabular}{|c|c|}
  \hline
  Интервалы&Рост\\
  \hline
  $\bigg(0;(\dfrac{\alpha sA}{\lambda+\mu})^{\frac{1}{1-\alpha}}\bigg)$&Ускоренный рост\\
  \hline
  $\bigg((\dfrac{\alpha sA}{\lambda+\mu})^{\frac{1}{1-\alpha}};(\dfrac{sA}{\lambda+\mu})^{\frac{1}{1-\alpha}}\bigg)$&Насыщенный рост\\
  \hline
  $\bigg((\dfrac{sA}{\lambda+\mu})^{\frac{1}{1-\alpha}};+\infty\bigg)$&Падение\\
  \hline
\end{tabular}
\end{center}
\textbf{Конечно-разностное представление}:
$k(t+\Delta)=k(t)+\Delta t(-(\lambda+\mu)k(t)+sAk(t)^{\alpha})$

\subsection{SIR модель}
Пусть $S(t)$ - число восприимчивых к инфекции\\
$I(t)$ - число инфицированных\\
$R(t)$ - число переболевших инфекцией\\
$N$ - число популяции\\
$\beta$ - коэффициент интенсивности контактов \\
$\gamma$ - коэффициент интенсивности выздоровления
\begin{align}
  &\dfrac{d S}{dt}=\dfrac{-\beta IS}{N}\notag\\
  &\dfrac{d I}{dt}=\dfrac{\beta IS}{N}-\gamma I\\
  &\dfrac{d R}{dt}=\gamma I\notag
\end{align}

\subsection{SEIRD модель}
$E(t)$ - число носителей заболевания\\
$D$ - число умерших\\
$\mu$ - уровень смертности\\
$\delta=\dfrac{1}{\text{ср.инк.период}}$
\begin{align}
  &\dfrac{d S}{dt}=\dfrac{-\beta IS}{N}\notag\\
  &\dfrac{d E}{dt}=\dfrac{\beta IS}{N}-\delta E\notag\\
  &\dfrac{d I}{dt}=\delta E-\gamma I-\mu I\\
  &\dfrac{d R}{dt}=\gamma I\notag\\
  &\dfrac{d D}{dt}=\mu I\notag
\end{align}

\subsection{Модель Лотки-Вольтерра}
\begin{equation}
  \begin{cases}
    \dot{x}=ax-bxy\\
    \dot{y}=-cy+dxy
  \end{cases}
\end{equation}
$x(t)$ - число жертв\\
$y(t)$ - число хищников\\
$a$ - коэффициент рождаемости жертв\\
$b$ - коэффициент убыли жертв\\
$c$ - коэффициент убыли хищников \\
$d$ - коэффициент рождаемости хищников\\
Первой стационарной точкой является $(0,0)$. Возьмем из системы линейную часть и составим матрицу.
\begin{equation}
  \begin{vmatrix}
  a-\lambda&0\\0&-c-\lambda\;
\end{vmatrix}
\end{equation}
Решая данной характеристическое уравнение получим $\lambda_1=a, \lambda_2=-c\Rightarrow$ данная точка является седлом.
Теперь приравняем правые части системы к 0 и решим ее. Получим вторую стационарную точку $\overline{x}=\dfrac{c}{d},
\overline{y}=\dfrac{a}{b}$. Построим матрицу Якоби, подставив $\overline{x},\overline{y}$.
\begin{equation}
  \begin{pmatrix}
    0&-\dfrac{bc}{d}\\-\dfrac{ad}{b}&0
  \end{pmatrix}
\end{equation}
Решая характеристическое уравнение $\lambda^2+ac=0$, получаем два мнимых корня, что свидетельствует о том что данная
стационарная точка является центром.

\subsection{Модель взаимодействия двух конкурирующих видов}
$x_1$ - количество особей первого типа
$x_2$ - количество особей второго типа
\begin{equation}
  \begin{cases}
    \dot{x}_1=a_1x_1-b_{11}x_1^2-b_{12}x_1x_2\\
    \dot{x}_2=a_2x_2-b_{21}x_1x_2-b_{22}x_2^2
  \end{cases}
\end{equation}
Приравняем к 0 правые части системы.
\begin{equation}
  \begin{cases}
    x_1(a_1-b_{11}x_1-b_{12}x_2)=0\\
    x_2(a_2-b_{21}x_1-b_{22}x_2)=0
  \end{cases}
\end{equation}
Получим 4 стационарные точки.
\begin{gather}
  \begin{cases}
    x_1=0\\
    x_2=0
  \end{cases};
  \begin{cases}
    x_1=0\\
    x_2=\dfrac{a_2}{b_{22}}
  \end{cases};\\
  \begin{cases}
    x_1=\dfrac{a_1}{b_{11}}\\
    x_2=0
  \end{cases};
  \begin{cases}
    x_1=\dfrac{a_2b_{12}-a_1b_{22}}{b_{12}b_{21}-b_{22}b_{11}}\\
    x_2=\dfrac{a_1b_{21}-a_2b_{11}}{b_{12}b_{21}-b_{22}b_{11}}
  \end{cases}\notag
\end{gather}
% \end{document}

% \documentclass[12pt]{article}
%  \usepackage[margin=1in]{geometry}
% \usepackage{amsmath, amsthm, amssymb, amsfonts, enumitem, fancyhdr, color, comment, graphicx, environ, fancyhdr}
% \pagestyle{fancy}
% \usepackage{cmap}
% \usepackage[T2A]{fontenc}
% \usepackage[utf8]{inputenc}
% \usepackage[russian, english]{babel}
% \usepackage{caption}
%
% \newtheorem*{exersize}{Упражнение}
% \newtheorem*{lemma}{Лемма}
% \newtheorem*{consequence}{Следствие}
% \newtheorem{theorem}{Theorem}[section]
% \newtheorem*{statement}{Утверждение}
% \newtheorem{property}{Свойство}
% \newtheorem*{fact}{Факт}
%
% \theoremstyle{definition}
% \newtheorem{definition}{Определение}[section]
% \newtheorem*{problem}{Задача}
% \newtheorem*{example}{Пример}
%
% \theoremstyle{remark}
% \newtheorem*{note}{Замечание}
% \setlength{\parindent}{0pt}
% \setlength{\headheight}{35pt}
%
%
% \begin{document}
\section{Бифуркации динамических систем}
\begin{definition}
Бифуркация - качественное изменение фазового портрета при изменении параметров системы.
\end{definition}
\begin{definition}
  Точка бифуркации - критическое состояние системы, при котором система становится неустойчивой относительно флуктуаций и возникает неопределённость: станет ли состояние системы хаотическим или она перейдёт на новый, более дифференцированный и высокий уровень упорядоченности.
\end{definition}
\begin{theorem}{Бифуркация Хопфа}\\
  Пусть есть система дифференциальных уравнений
  \begin{equation}
    \begin{cases}
      \dot{x}_1=X_1(x_1,...,x_n,\mu) \qquad X_1(0,...,0,\mu)=0\\
      \dot{x}_2=X_2(x_1,...,x_n,\mu) \qquad X_2(0,...,0,\mu)=0
    \end{cases}
  \end{equation}
$\lambda_1(\mu_0),\lambda_2(\mu_0)$ - чисто мнимые корни. Точка $(0,0)$ - асимптотически устойчива при $\mu_0$ и
$\dfrac{\partial}{\partial \mu}\{Re(\lambda_i(\mu))|_{\mu=\mu_0}\}>0$.\\
Тогда \begin{enumerate}
  \item $\mu_0$ - точка бифуркации
  \item $\exists$ интервал $\mu\in(\mu_1,\mu_0)$ такой, что $(0;0)$ - устойчивый фокус
  \item $\exists$ интервал $\mu\in(\mu_0,\mu_2)$ такой, что $(0;0)$ - неустойчивый фокус, окруженный предельным циклом
\end{enumerate}
\end{theorem}

\subsection{Аттрактор Лоренца}

Модель Лоренца является реальным физическим примером динамических систем с хаотическим поведением, в отличие от различных искусственно сконструированных отображений (преобразование пекаря, отображение Фейгенбаума).
\begin{definition}
Динамический хаос или детерминированный хаос - явление в теории динамических систем, при котором поведение нелинейной системы выглядит случайным, хотя оно определяется детерминистическими законами.
\end{definition}
Аттрактор Лоренца возникает в следующих физических вопросах:
\begin{itemize}
  \item конвекция в замкнутой петле;
\item вращение водяного колеса;
\item модель одномодового лазера;
\item диссипативный гармонический осциллятор с инерционной нелинейностью.
\end{itemize}
Аттрактор описывается следующей системой дифференциальных уравнений.
\begin{equation}
  \begin{cases}
    \dot{x}=a(y-x)\\
    \dot{y}=x(r-z)-y\\
    \dot{z}=-bz+xy \qquad a,r,b>0
  \end{cases}
\end{equation}
$r$ - управляющий переменный параметр;
\begin{itemize}
\item $(0<r<1)$ - одна критическая точка;
\item $r\to 1$ - критическое замедление;
\item $r=1.345$ - узлы переходят в фокусы;
\item $r\approx 13.927$ - если траектория выходит из начала координат, то, совершив полный оборот вокруг одной из устойчивых точек, она вернется обратно в начальную точку — возникают две гомоклинические петли. Понятие гомоклинической траектории означает, что она выходит и приходит в одно и то же положение равновесия;
\item $r>13.927$ -  в зависимости от направления траектория приходит в одну из двух устойчивых точек. Гомоклинические петли перерождаются в неустойчивые предельные циклы, также возникает семейство сложно устроенных траекторий, не являющееся аттрактором, а скорее наоборот, отталкивающее от себя траектории. Иногда по аналогии эта структура называется «странным репеллером»;
\item $r>24$ - хаос.
\end{itemize}
При $r=16$ на рисунке 45 видно образование неустойчивых предельных циклов.
\begin{center}
  \begin{tabular}{c}
  \includegraphics[scale=0.4]{"images/lorenz_1.png"}\\
  Рисунок 45 - Аттрактор Лоренца $r=16$
\end{tabular}
\end{center}
При $r=16$ на рисунке 46 виден сам Аттрактор Лоренца и его хаотическое поведение на графиках по времени.
\begin{center}
  \begin{tabular}{c}
  \includegraphics[scale=0.4]{"images/lorenz_2.png"}\\
  Рисунок 46 - Аттрактор Лоренца $r=23$
\end{tabular}
\end{center}

\subsection{Аттрактор Рикитаки}
\begin{align}
  &\dfrac{d x}{dt} = -\mu x+yz \notag\\
  &\dfrac{d y}{dt} = (z-a)x-\mu y\\
  &\dfrac{d z}{dt} = 1-xy\notag
\end{align}
На рисунках ниже представлен численный анализ модели при различных параметрах $\mu$ и $a$.
При параметрах $\mu=0.2,a=0.5$ получаем спиральный устойчивый фокус сходящийся в направлении плоскости $XY$ (рис. 47).

\begin{center}
  \begin{tabular}{c}
  \includegraphics[scale=0.4]{images/atrRik_1.png}\\
  Рисунок 47 - Аттрактор Рикитаки $\mu=0.2,a=0.5$
\end{tabular}
\end{center}

При параметрах $\mu>=1,a=0.9$ получаем седло (рис. 48).
\begin{center}
  \begin{tabular}{c}
  \includegraphics[scale=0.35]{images/atrRik_2.png}\\
  Рисунок 48 - Аттрактор Рикитаки $\mu>=1,a=0.9$
\end{tabular}
\end{center}
При параметрах $\mu=0.498,a=0.1$ получаем график, где видно и седло и спиральный фокус (рис. 49).

\begin{center}
  \begin{tabular}{c}
  \includegraphics[scale=0.4]{images/atrRik_3.png}\\
  Рисунок 49 - Аттрактор Рикитаки $\mu=0.498,a=0.1$
\end{tabular}
\end{center}

Далее представлены графики векторного поля по осям. На первом рисунке виден спиральный устой фокус в плоскости $XY$, а на второй в плоскости $XZ$ видно седло и устойчивый фокус одновременно (рис. 50, 51).

\begin{center}
  \begin{tabular}{c}
  \includegraphics[scale=0.4]{images/atrRik_4.png}\\
  Рисунок 50 - Векторное поле в $XY$
  \end{tabular}
  \end{center}
  \begin{center}
    \begin{tabular}{c}
  \includegraphics[scale=0.4]{images/atrRik_5.png}\\
  Рисунок 51 - Векторное поле в $XZ$
\end{tabular}
\end{center}

% \end{document}


\section{Оптимизационные модели}
\subsection{Задача о производстве мебели}

\textbf{Условие}

Есть $N$ типов $A_i$ продукции, $M$ операций $B_j$ для производства продукции, $c_i$ -- прибыль за единицу $i$-той произведенной продукции, $a_{ij}$ -- количество времени, необходимого на операцию $j$ при производстве продукции $i$. $b_j$ -- ограничение на количество времени по каждой операции.

\textbf{Решение}
\begin{center}
    \begin{tabular}{ |c|c|c|c|c|c|c|}
         \hline
               & $B_1$   & $B_2$   & ...     & $B_M$   & Прибыль & Количество \\
         \hline
         $A_1$ & $a_{11}$& $a_{12}$& ...     & $a_{1M}$& $p_1$   & $x_1$      \\
         $A_2$ & $a_{21}$& $a_{22}$& ...     & $a_{2M}$& $p_2$   & $x_2$      \\
         ...   & ...     & ...     & ...     & ...     & ...     & ...        \\
         $A_N$ & $a_{N1}$& $a_{N2}$& ...     & $a_{NM}$& $p_N$   & $x_N$      \\
         \hline
         Огр   & $b_1$   & $b_2$   & ...     & $b_M$   &         &            \\
         \hline
    \end{tabular}
\end{center}

$\forall i: x_i \geqslant 0, \; x_i \in \mathbb{Z}$ -- количество произведенной продукции каждого типа
$$f(X) = p_1 x_1 + p_2 x_2 + ... + p_N x_N \rightarrow \max$$
$$a_{11} x_1 + a_{21} x_2 + ... + a_{N1} x_N \leqslant b_1$$
$$a_{1j} x_1 + a_{2j} x_2 + ... + b_{Nj} x_N \leqslant b_j$$
$$a_{1M} x_1 + a_{2M} x_2 + ... + b_{NM} x_N \leqslant b_M$$

$$\begin{cases}
    f(X) = \sum_{i=1}^N p_i x_i \rightarrow \max \\
    \sum_{i=1}^N a_{ij} x_i \leqslant b_j \\
    \forall i: x_i \geqslant 0, \; x_i \in \mathbb{Z}
\end{cases}$$

В данной конкретной задаче математическая модель выглядит следующим образом:
$$\begin{cases}
    f(X) = 16 x_1 + 30 x_2 + 40 x_3 + 42 x_4 + 32 x_5 \rightarrow \max \\
    8 x_1 + 6 x_2 + 9 x_3 + 9 x_4 + 12 x_5 \leqslant 320 \\
    12 x_1 + 10 x_2 + 15 x_3 + 12 x_4 + 8 x_5 \leqslant 400 \\
    4 x_1 + 3 x_2 + 5 x_3 + 4 x_4 + 4 x_5 \leqslant 270 \\
    \forall i: x_i \geqslant 0, \; x_i \in \mathbb{Z}
\end{cases}$$

\section*{a}
Какая комбинация изделий должна быть произведена в это время, чтобы максимизировать прибыль? Какой будет общая прибыль?
Для решения этой задачи был использован Excel.

\begin{center}
  \begin{tabular}{c}
\includegraphics[scale=0.45]{images/furniture1.png}\\
Рисунок 52 - Пункт $a$
\end{tabular}
\end{center}

Общая прибыль равна 1408 единицам.

\section*{b}
Выгодно ли производить все изделия? Если имеется изделие, которое не выгодно производить, что нужно изменить, чтобы его производство стало выгодным?

Не все изделия выгодно производить. Например, изделия <<Стул>>, <<Стол>> и <<Бюро>> в оптимальном плане не производятся вообще. Это значит, что их производство невыгодно. Чтобы их производство стало выгодным, необходимо, к примеру, увеличить их стоимость или изменить технологию производства. Например, уменьшив время на шлифовку изделия <<Бюро>> на четыре единицы, получаем увеличение целевой функции на 24 единицы, а также изделие <<Бюро>> становится наиболее выгодным для производства.

\begin{center}
  \begin{tabular}{c}
\includegraphics[scale=0.45]{images/furniture5.png}\\
Рисунок 53 - Пункт $b$
\end{tabular}
\end{center}
\section*{c}
Можно ли изменить что-то в технологии или в ценах так, чтобы все изделия стали выгодными? Исследуйте это. Опишите результаты.

Чтобы их производство всех изделий стало выгодным, необходимо, к примеру, увеличить их стоимость. Чтобы понять, на сколько надо увеличить стоимость изделий, проведем анализ чувствительности с помощью Excel.

\begin{center}
  \begin{tabular}{c}
\includegraphics[scale=0.45]{images/furniture3.png}\\
Рисунок 54 - Пункт $c$
\end{tabular}
\end{center}

Reduced Cost (<<Приведенная цена>> в русифицированной версии Excel) показывает, на сколько нужно увеличить стоимость изделий, чтобы они вошли в оптимальный план. Если увеличить стоимость невыгодных изделий, то получим следующий результат:

\begin{center}
  \begin{tabular}{c}
\includegraphics[scale=0.45]{images/furniture4.png}\\
Рисунок 55 - Пункт $c$
\end{tabular}
\end{center}

Практически все изделия стало выгодно производить. Общая прибыль увеличилась на пять единиц.


\section*{d}
Допустим, что Вы можете установить 100 сверхурочных минут, но для только одной из основных операций? На какую операцию стоит выделить это время? Сколько при этом получится прибыли? Подтвердите все ваши ответы вычислениями.

Если посмотреть на решение пункта $a$, то можно увидеть, что для всех операций осталось свободное время, кроме шлифовки. Таким образом, добавив время к обрезке или сборке, мы не увеличим количество произведенной продукции, так как для шлифовки не будет хватать времени. Стоит выделить сверхурочное время для шлифовки. Также из решения пункта $c$ видно, что Shadow price (<<теневая цена>>) для шлифовки наибольшая. То есть доход больше всего увеличится при увеличении времени на шлифовку. Для решения этой задачи был использован Excel.

\begin{center}
  \begin{tabular}{c}
\includegraphics[scale=0.45]{images/furniture2.png}\\
Рисунок 56 - Пункт $d$
\end{tabular}
\end{center}

При установлении 100 сверхурочных минут к шлифовке общая прибыль увеличилась на 152 единицы.

\subsection{Задача о назначениях}
\textbf{Условие}

Необходимо составить 6 пар-команд для выполнения некоторой работы. Индекс совместности варьируется от 20 (выраженная враждебность) до 1 (дружеские отношения) и представлен в таблице 1 на рисунке 57. Необходимо составить команды, минимизировав суммарную совместность. Какой минимально возможный для всех пар индекс совместимости можно обеспечить и как при этом изменится суммарная совместимость.

\textbf{Решение}

Для решения данной задачи также воспользуемся в $Excel$ поиском решения. Чтобы просто составить пары-команды, необходимо задать ограничения на сумму по столбцами и строкам равную 1, так как один человек может образовать пару только с одним другим человеком. Также надо указать, что работаем с бинарными переменными. Целевая функция для данной задачи это сумма произведения элементов матрицы совместности на элементы матрицы назначений. Решение представлено в таблице 2 на рисунке 57.\\

Чтобы ответить на второй вопрос какой минимальный индекс совместности можно выбрать, надо дополнительно вручную задавать ограничения на максимальный индекс совместности при выборе пар. Таким индексом здесь является 6. При выборе меньшего индекса уменьшается значение целевой функции, а следовательно, такое решение брать не оптимально.
\begin{center}
  \begin{tabular}{c}
\includegraphics[scale=0.5]{images/assign.png}\\
Рисунок 57 - Задача о назначениях
\end{tabular}
\end{center}

\subsection{Задача про аудиторов}

\textbf{Условие}

Менеджер аудиторской фирмы должен распределить
аудиторов для работы на следующий месяц. Аудиторы
различаются по квалификации и опыту работы. Прежде чем
приступить к аудиту конкретной фирмы они должны затратить
определенное время на подготовку и консультации. В данный
момент имеются заявки от 10 клиентов. Менеджер – координатор,
учитывая опыт работ аудиторов каждой конторы, оценил время,
необходимое «среднему» аудитора каждой конторы для подготовки
к аудиту конкретного клиента

\textbf{Решение}

Целевой функцией в этой задаче также является сумма произведений элементов двух таблиц. Целевая функция минимизируется. Также необходимы ограничения, которые позволяют распределить всех сотрудников без остатка, то есть сумма по столбцам и строкам новой таблицы минус заявки и число сотрудников соответственно из исходной таблицы с данными (рис. 58). Там где имеются пропуски, чтобы не рассматривать данные ячейки поставим большие значения, так как функция минимизируется.
\begin{center}
  \begin{tabular}{c}
\includegraphics[scale=0.5]{images/audit (1).png}\\
Рисунок 58 - Задача про аудиторов п.1
\end{tabular}
\end{center}

Решение пункта два аналогично, только теперь для столбцов не будем вычитать данные исходной таблицы, а найдем решение только для ограничений по строкам. Далее будем вычитать поочередно из каждого элемента столбца получившеюся сумму всего столбца, тем самым получим вариации для распределения аудиторов (рис. 59).
\begin{center}
  \begin{tabular}{c}
\includegraphics[scale=0.5]{images/audit (2).png}\\
Рисунок 59 - Задача про аудиторов п.2
\end{tabular}
\end{center}

\subsection{Модель потребительского выбора}
В общем случае постановка задачи выглядит следующим образом.
\begin{equation}
  \begin{cases}
u(x_1,...,x_n)\to\max\\
\sum_{i=1}^np_ix_i\leq I\\
x_i\geq 0
  \end{cases}
\end{equation}
\subsubsection{Модель Стоуна}
\begin{equation}
  u=\prod_{i=1}^n(x_i-\alpha_i)^{\beta_i}\to\max
\end{equation}
где $\alpha_i\leqslant 0, x_i\leqslant \alpha_i$,$0<\beta_i<1$ - степень важности благ.
Составив функцию Лагранжа и выполнив несложные преобразования получим следующее выражение.
\begin{equation}
x_k=\alpha_k+\dfrac{1}{p_k}\dfrac{(I-\sum_i\alpha_i p_i)\beta_k}{\sum_i\beta_i}
\end{equation}
\end{document}
