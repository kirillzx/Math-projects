% \documentclass[12pt]{article}
%  \usepackage[margin=1in]{geometry}
% \usepackage{amsmath, amsthm, amssymb, amsfonts, enumitem, fancyhdr, color, comment, graphicx, environ, fancyhdr}
% \pagestyle{fancy}
% \usepackage{cmap}
% \usepackage[T2A]{fontenc}
% \usepackage[utf8]{inputenc}
% \usepackage[russian, english]{babel}
% \usepackage{caption}
%
% \newtheorem*{exersize}{Упражнение}
% \newtheorem*{lemma}{Лемма}
% \newtheorem*{consequence}{Следствие}
% \newtheorem{theorem}{Theorem}[section]
% \newtheorem*{statement}{Утверждение}
% \newtheorem{property}{Свойство}
% \newtheorem*{fact}{Факт}
%
% \theoremstyle{definition}
% \newtheorem{definition}{Определение}[section]
% \newtheorem*{problem}{Задача}
% \newtheorem*{example}{Пример}
%
% \theoremstyle{remark}
% \newtheorem*{note}{Замечание}
% \setlength{\parindent}{0pt}
% \setlength{\headheight}{35pt}
%
%
% \begin{document}
\section{Бифуркации динамических систем}
\begin{definition}
Бифуркация - качественное изменение фазового портрета при изменении параметров системы.
\end{definition}
\begin{definition}
  Точка бифуркации - критическое состояние системы, при котором система становится неустойчивой относительно флуктуаций и возникает неопределённость: станет ли состояние системы хаотическим или она перейдёт на новый, более дифференцированный и высокий уровень упорядоченности.
\end{definition}
\begin{theorem}{Бифуркация Хопфа}\\
  Пусть есть система дифференциальных уравнений
  \begin{equation}
    \begin{cases}
      \dot{x}_1=X_1(x_1,...,x_n,\mu) \qquad X_1(0,...,0,\mu)=0\\
      \dot{x}_2=X_2(x_1,...,x_n,\mu) \qquad X_2(0,...,0,\mu)=0
    \end{cases}
  \end{equation}
$\lambda_1(\mu_0),\lambda_2(\mu_0)$ - чисто мнимые корни. Точка $(0,0)$ - асимптотически устойчива при $\mu_0$ и
$\dfrac{\partial}{\partial \mu}\{Re(\lambda_i(\mu))|_{\mu=\mu_0}\}>0$.\\
Тогда \begin{enumerate}
  \item $\mu_0$ - точка бифуркации
  \item $\exists$ интервал $\mu\in(\mu_1,\mu_0)$ такой, что $(0;0)$ - устойчивый фокус
  \item $\exists$ интервал $\mu\in(\mu_0,\mu_2)$ такой, что $(0;0)$ - неустойчивый фокус, окруженный предельным циклом
\end{enumerate}
\end{theorem}

\subsection{Аттрактор Лоренца}

Модель Лоренца является реальным физическим примером динамических систем с хаотическим поведением, в отличие от различных искусственно сконструированных отображений (преобразование пекаря, отображение Фейгенбаума).
\begin{definition}
Динамический хаос или детерминированный хаос - явление в теории динамических систем, при котором поведение нелинейной системы выглядит случайным, хотя оно определяется детерминистическими законами.
\end{definition}
Аттрактор Лоренца возникает в следующих физических вопросах:
\begin{itemize}
  \item конвекция в замкнутой петле;
\item вращение водяного колеса;
\item модель одномодового лазера;
\item диссипативный гармонический осциллятор с инерционной нелинейностью.
\end{itemize}
Аттрактор описывается следующей системой дифференциальных уравнений.
\begin{equation}
  \begin{cases}
    \dot{x}=a(y-x)\\
    \dot{y}=x(r-z)-y\\
    \dot{z}=-bz+xy \qquad a,r,b>0
  \end{cases}
\end{equation}
$r$ - управляющий переменный параметр;
\begin{itemize}
\item $(0<r<1)$ - одна критическая точка;
\item $r\to 1$ - критическое замедление;
\item $r=1.345$ - узлы переходят в фокусы;
\item $r\approx 13.927$ - если траектория выходит из начала координат, то, совершив полный оборот вокруг одной из устойчивых точек, она вернется обратно в начальную точку — возникают две гомоклинические петли. Понятие гомоклинической траектории означает, что она выходит и приходит в одно и то же положение равновесия;
\item $r>13.927$ -  в зависимости от направления траектория приходит в одну из двух устойчивых точек. Гомоклинические петли перерождаются в неустойчивые предельные циклы, также возникает семейство сложно устроенных траекторий, не являющееся аттрактором, а скорее наоборот, отталкивающее от себя траектории. Иногда по аналогии эта структура называется «странным репеллером»;
\item $r>24$ - хаос.
\end{itemize}
При $r=16$ на рисунке 45 видно образование неустойчивых предельных циклов.
\begin{center}
  \begin{tabular}{c}
  \includegraphics[scale=0.4]{"images/lorenz_1.png"}\\
  Рисунок 45 - Аттрактор Лоренца $r=16$
\end{tabular}
\end{center}
При $r=16$ на рисунке 46 виден сам Аттрактор Лоренца и его хаотическое поведение на графиках по времени.
\begin{center}
  \begin{tabular}{c}
  \includegraphics[scale=0.4]{"images/lorenz_2.png"}\\
  Рисунок 46 - Аттрактор Лоренца $r=23$
\end{tabular}
\end{center}

\subsection{Аттрактор Рикитаки}
\begin{align}
  &\dfrac{d x}{dt} = -\mu x+yz \notag\\
  &\dfrac{d y}{dt} = (z-a)x-\mu y\\
  &\dfrac{d z}{dt} = 1-xy\notag
\end{align}
На рисунках ниже представлен численный анализ модели при различных параметрах $\mu$ и $a$.
При параметрах $\mu=0.2,a=0.5$ получаем спиральный устойчивый фокус сходящийся в направлении плоскости $XY$ (рис. 47).

\begin{center}
  \begin{tabular}{c}
  \includegraphics[scale=0.4]{images/atrRik_1.png}\\
  Рисунок 47 - Аттрактор Рикитаки $\mu=0.2,a=0.5$
\end{tabular}
\end{center}

При параметрах $\mu>=1,a=0.9$ получаем седло (рис. 48).
\begin{center}
  \begin{tabular}{c}
  \includegraphics[scale=0.35]{images/atrRik_2.png}\\
  Рисунок 48 - Аттрактор Рикитаки $\mu>=1,a=0.9$
\end{tabular}
\end{center}
При параметрах $\mu=0.498,a=0.1$ получаем график, где видно и седло и спиральный фокус (рис. 49).

\begin{center}
  \begin{tabular}{c}
  \includegraphics[scale=0.4]{images/atrRik_3.png}\\
  Рисунок 49 - Аттрактор Рикитаки $\mu=0.498,a=0.1$
\end{tabular}
\end{center}

Далее представлены графики векторного поля по осям. На первом рисунке виден спиральный устой фокус в плоскости $XY$, а на второй в плоскости $XZ$ видно седло и устойчивый фокус одновременно (рис. 50, 51).

\begin{center}
  \begin{tabular}{c}
  \includegraphics[scale=0.4]{images/atrRik_4.png}\\
  Рисунок 50 - Векторное поле в $XY$
  \end{tabular}
  \end{center}
  \begin{center}
    \begin{tabular}{c}
  \includegraphics[scale=0.4]{images/atrRik_5.png}\\
  Рисунок 51 - Векторное поле в $XZ$
\end{tabular}
\end{center}

% \end{document}
