% \documentclass[12pt]{article}
%  \usepackage[margin=1in]{geometry}
% \usepackage{amsmath, amsthm, amssymb, amsfonts, enumitem, fancyhdr, color, comment, graphicx, environ, fancyhdr}
% \pagestyle{fancy}
% \usepackage{cmap}
% \usepackage[T2A]{fontenc}
% \usepackage[utf8]{inputenc}
% \usepackage[russian, english]{babel}
% \usepackage{caption}
%
% \newtheorem*{exersize}{Упражнение}
% \newtheorem*{lemma}{Лемма}
% \newtheorem*{consequence}{Следствие}
% \newtheorem{theorem}{Theorem}[section]
% \newtheorem*{statement}{Утверждение}
% \newtheorem{property}{Свойство}
% \newtheorem*{fact}{Факт}
%
% \theoremstyle{definition}
% \newtheorem{definition}{Definition}[section]
% \newtheorem*{problem}{Задача}
% \newtheorem*{example}{Пример}
%
% \theoremstyle{remark}
% \newtheorem*{note}{Замечание}
% \setlength{\parindent}{0pt}
% \setlength{\headheight}{35pt}
%
%
% \begin{document}
\section{Бифуркации динамических систем}
\begin{definition}
Бифуркация - качественное изменение фазового портрета при изменении параметров системы.
\end{definition}
\begin{theorem}{Бифуркация Хопфа}\\
  Пусть есть система дифференциальных уравнений
  \begin{equation}
    \begin{cases}
      \dot{x}_1=X_1(x_1,...,x_n,\mu) \qquad X_1(0,...,0,\mu)=0\\
      \dot{x}_2=X_2(x_1,...,x_n,\mu) \qquad X_2(0,...,0,\mu)=0
    \end{cases}
  \end{equation}
$\lambda_1(\mu_0),\lambda_2(\mu_0)$ - чисто мнимые корни. Точка $(0,0)$ - асимптотически устойчива при $\mu_0$ и
$\dfrac{\partial}{\partial \mu}\{Re(\lambda_i(\mu))|_{\mu=\mu_0}\}>0$.\\
Тогда \begin{enumerate}
  \item $\mu_0$ - точка бифуркации
  \item $\exists$ интервал $\mu\in(\mu_1,\mu_0)$ такой, что $(0;0)$ - устойчивый фокус
  \item $\exists$ интервал $\mu\in(\mu_0,\mu_2)$ такой, что $(0;0)$ - неустойчивый фокус, окруженный предельным циклом
\end{enumerate}
\end{theorem}

\subsection{Аттрактор Лоренца}
\begin{equation}
  \begin{cases}
    \dot{x}=-ax+ay\\
    \dot{y}=rx-y-xz\\
    \dot{z}=-bz+xy \qquad a,r,b>0
  \end{cases}
\end{equation}
$r$ - управляющий переменный параметр\\
$(0<r<1)$ - одна критическая точка\\
$r\to 1$ - критическое замедление\\
$r=1.345$ - узлы переходят в фокусы\\
$r>24$ - хаос

\subsection{Маятник Фуко}
Пусть $L$ - длина нити маятника\\
$\omega$ - угловая скорость\\
$g$ - ускорение свободного падения\\
$x,y$ - координаты\\
$v_x, v_y$ - скорости
\begin{align}
  &\dfrac{d v_x}{dt} = 2v_y\omega+\omega^2x-g\dfrac{x}{L}\notag\\
  &\dfrac{d v_y}{dt} = -2v_x\omega+\omega^2y-g\dfrac{y}{L}\\
  &\dfrac{d x}{dt} = v_x\notag\\
  &\dfrac{d y}{dt} = v_y\notag
\end{align}
\subsection{Аттрактор Рикитаки}
\begin{align}
  &\dfrac{d x}{dt} = -\mu x+yz \notag\\
  &\dfrac{d y}{dt} = (z-a)x-\mu y\\
  &\dfrac{d z}{dt} = 1-xy\notag
\end{align}
На рисунках ниже представлен численный анализ модели при различных параметрах $\mu$ и $a$.
При параметрах $\mu=0.2,a=0.5$ получаем спиральный устойчивый фокус сходящийся в направлении плоскости $XY$
\begin{center}
  \includegraphics[scale=0.6]{images/atrRik_1.png}
\end{center}
При параметрах $\mu>=1,a=0.9$ получаем седло.
\begin{center}
  \includegraphics[scale=0.4]{images/atrRik_2.png}
\end{center}
При параметрах $\mu=0.498,a=0.1$ получаем график, где видно и седло и спиральный фокус.
\begin{center}
  \includegraphics[scale=0.4]{images/atrRik_3.png}
\end{center}
Далее представлены графики векторного поля по осям. На первом рисунке виден спиральный устой фокус в плоскости $XY$, а на второй в плоскости $XZ$ видно седло и устойчивый фокус одновременно.
\begin{center}
  \includegraphics[scale=0.4]{images/atrRik_4.png}
  \includegraphics[scale=0.4]{images/atrRik_5.png}
\end{center}

% \end{document}
